\section{Statistics}

\subsection{Correllation}
Assume an inner product space.

\subsection{Linear regression}

Assume that reality can me modelled by a model like this one: 

$$ \vec{y} = \alpha \vec{x} + \vec{e} $$

\subsection{Principal component analysis}


\subsection{Tests}
Often, statistical tests work \emph{the wrong way round}. You have your hypothesis, like "these groups are different". Then you state the opposite, the \emph{null-hypothesis}. Then you determine how likely your data is under the null-hypothesis - the \emph{p-value}. And than you hope that the p-value is very low.

\paragraph{Students T-Test} compares two averages (means) and tells you if they are different from each other. The null hypothesis is that all samples come from the same distribution. That means that under $H_0$, the two means you obtained should be similar. So the interpretation goes like this: 
\begin{itemize}
    \item low p-value: the chance that your result would have happend under $H_0$ is low. It is very unlikely that the two means you have obtained actually come from the same distribution. 
    \item high p-value: It is quite possible that both your groups come from the same distribution. 
\end{itemize}
Note that there are two experimental setups that can use t-tests:
\begin{itemize}
    \item unpaired: you have two groups, with no person in both groups. Example: Test one group with medication and another with placebo. 
    \item paired: every person is first in the first and then in the second group. Example: test patients before and after treatment. 
\end{itemize}