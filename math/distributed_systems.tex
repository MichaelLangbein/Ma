\section{Distributed systems}

We implement distributed systems when our system becomes too big to be run from a single computer. Distribution always makes things more complex, so there is no reason to implement it when you don't have to. But in almost every case, you eventually will. Most of the following text is based on \href{https://disco.ethz.ch/courses/podc_allstars/lecture/chapter17.pdf}{this lecture series}.

\subsection{Model}
\begin{itemize}
    \item \textbf{Communication graph} $G$: a directed, two-way graph
      \begin{itemize}
        \item nodes: \nodes{G}
        \item edges: \edges{G}
        \item $\forall (u, v) \in \edges{G}: (v, u) \in \edges{G}$
      \end{itemize}
    \item \textbf{Subgraph} $U_G$:
      \begin{itemize}
        \item nodes: \nodes{U}
        \item edges: $\edges{U} = \{(u, v) | u \in \nodes{U} \land v \in \nodes{U}\}$
      \end{itemize}
    \item \textbf{Inedge border} of $U_G$: $\{(u, v) | u \notin \nodes{U} \land v \in \nodes{U}\}$
    \item \textbf{System} $\mathcal{G}$:
      \begin{itemize}
        \item a communication graph $G$
        \item a program $A$ for each node $n$ of $G$ (we say: $n$ runs $A$) (in the original paper, a program was called a \emph{device})
        \item input to each node $n$ of $G$
        \item behaviour $\mathcal{B_G}$
      \end{itemize}
\end{itemize}

\subsection{Axioms}
\paragraph{The locality axiom} states that a subsystems behaviour is only influenced by its inedge border.
\paragraph{The fault axiom} states that there exists a (malicious) program $F$ that can mimic other programs, so that it behaves towards node $a$ like $X$, and towards $b$ like $Y$. 

\subsection{Theorems}

\subsubsection{Coverings}
Define the \emph{neighborhood} of node $u$ as $\text{nbh}_u = \{ v \in \nodes{G} | (v, u) \in \edges{G} \}$. The neighborhood of $u$ is simply the inedge border of a one-node-graph $U := \{u\}$. A graph $S$ \emph{covers} $G$ if there is a mapping $\phi$ from any node $s \in S$ to $u \in G$ that preservers neighborhood, that is: 
$$ \phi: \nodes{S} \to \nodes{G} $$
$$ \forall s \in \nodes{S}: \phi(s) = w \then \phi( \text{nbh}_s ) = \text{nbh}_w $$
Under such a mapping, $S$ looks locally like $G$. This is trivially proven. By the locality axiom, the behavior of a subsystem is only influenced by its inedge border. That means that the behaviour of the system on $u$ is only influenced by its neighborhood. Since every node $s \in \nodes{S}$ has a neighborhood identical to that of some $u \in \nodes{G}$, the behaviour of every node in $S$ is identical to that of some node in $G$.

In practice, we use this finding for the following strategy: Consider a big graph $S$. In $S$, we consider a subgraph $S'$. We might not know a lot about the behaviour of $S'$, but very likely, $G'$, which is covered by $S'$, is much simpler. Using the locality axiom we prove that $S'$ has to behave like $G'$.

\subsubsection{Byzantine agreement}
Imagine three byzantine generals laying siege to an enemy city. They have to decide if they want to attack the city or not. An attack can only be successful if they all attack together, so they must agree on attack or abstain together. Each general has his own information on whether an attack, even with all three generals involved, will be successful or not. Finally, there is a traitor among them, who will do everything he can to make them decide on the wrong strategy. The two loyal (a.k.a. \emph{correct}) generals follow some algorithm (like e.g. majority-vote, or seniority) for making their decision, while the traitor will do anything he wants.
For byzantine agreement to be possible, the following must hold:
\begin{enumerate}
    \item \emph{Agreement}: The two loyal generals decide upon the same value, that is, whether to attack or not.
    \item \emph{Validity}: The decision is made soundly. If both generals think its a good idea to attack, the agreement must be on 'attack'; if both think that it isn't, then the agreement must be on 'abstain'. If one general thinks an attack would be smart while the other doesn't, this rule doesn't hold, and only item 1 holds.
\end{enumerate}
In other words, if byzantine agreement were possible, it would mean that there is an one traitor cannot make the generals choose a bad strategy. However, we will proof that byzantine agreement is \emph{not} possible.

We can easily proof this. We have three nodes $u$,$v$,$w$. In order to achieve validity, a correct node must decide on its own value if another node supports that value.The third node might disagree, but that node could be a traitor. If correct node $u$ has input 0 and correct node $v$ has input 1, the byzantine node $w$ can fool them by telling $u$ that its value is 0 and simultaneously telling $v$ that its value is 1. This leads to $u$ and $v$ deciding on their own values, which results inviolating the agreement condition. Even if $u$ talks to $v$, and they figure out that they have different assumptions about $w$’s value, $u$ cannot distinguish whether $w$ or $v$ is a traitor.

But there is another, more general proof, which we can use for the following theorems as well. The general idea of the proof is as follows. 
\begin{enumerate}
    \item Assume byzantine agreement (B.A.) is possible.
    \item Consider $S$, a 6-n graph with inputs and programs, which we will construct in a cunning way. $S$ does not need to exhibit B.A. It only needs to be free of contradictions.
    \item Consider one flank $S_1$ of $S$. This flanks behaviour $B_1$ must be equal to that of a ceratain $G$, which exhibits B.A. and is covered by $S_1$. 
    \item Consider another, adjacent flank $S_2$ of $S$. This flanks behaviour $B_2$ must be equal to that of another $G$, and also be in accordance with $B_1$. 
    \item Consider a third, adjacent flank $S_3$ of $S$. This flanks behaviour $B_3$ must be equal to that of yet another $G$. However, $B_3$ will be a contradiction to $B_2$.
    \item Thus, if B.A. were possible, $S$ could not possibly exist. But it is obvious that $S$ can exist.
\end{enumerate}


\subsubsection{Week agreement}
\subsubsection{Byzantine fireing squad}
\subsubsection{Approximate agreement}
\subsubsection{Clock syncing}


\subsubsection{When timing does and does not matter}
Operations on a single machine have a total order, on distributed machines we can at least hope to acchieve a partial order.
Consider a transition function from one state to another. When is a transition function commutative? That would mean that the timing in which input arrives at a state machine does not matter. We know this is not always true. Just consider the state function $f(s, x) = sx + x$. 
$$ f : S \times X \to S$$
Theorem:
$$ f: \text{some property} \then f(f(s_0, x_0), x_1) = f(f(s_0, x_1), x_0)$$

\subsubsection{flp theorem}

\paragraph{System model} consists of:
\paragraph{In an asynchronous system, a server-failure can alter the outcome} This is
\paragraph{When you are at a bivalent state, you can delay a message such that you end up at a bivalent state again} This means that you can forever postpone the system from ever reaching a conclusion. 

\subsubsection{CAP-Theorem}


\subsection{Important alogrithms}

\subsubsection{one-, two- and tree-phase commit}

\subsubsection{paxos algorithm}

\subsubsection{raft algorithm}

\subsubsection{map reduce algorithm}
where in map reduce is the window of failure according to flp?


\subsection{Databases}

\subsubsection{ACID}
A transaction is a sequence of database operations that can be considered one independent, coherent unit of work. Database transaction should have the ACID-properties:
\begin{itemize}
    \item Atomicity
    \item Consistency
    \item Isolation
    \item Durability
\end{itemize}

If a distributed database provides the acid-properties, then it must chose consistency over availability according to the cap theorem. A available database cannot provide acid-transactions. 

\subsubsection{NoSQL Database types}
\includegraphics[scale=0.5]{images/cap_choice.png}


\subsubsection{Normalisation}
Normalisation is a way of structuring a relational database such that logical errors are minimized. 

\begin{lstlisting}[language=python]
from DB import DB


class DictOfSets:
    data = {}

    def addData(self, d):
        for row in d:
            if row[0] in self.data:
                self.data[row[0]].add(row[1])
            else:
                self.data[row[0]] = set([row[1]])

    def addVals(self, v):
        for row in v:
            if row[0] in self.data:
                self.data[row[0]].add(row[1])

    def addKeys(self, k):
        for key in k:
            if not key in self.data:
                self.data[key] = set()

        

class DBAbstr:
    host = '10.173.202.110'
    user = 'lesen'
    password = ''
    database = 'gkd_chemie'
    
    def doQuery(self, query):
        result = []
        with DB(self.host, self.user, self.password, self.database) as db:
            result = db.queryToArray(query)
        return result


class SetTable(DBAbstr):
    vals = []

    def __init__(self, tableName, key):
        self.tableName = tableName
        self.key = key

    def getVals(self):
        if self.vals:
            return self.vals
        query = "select {0} from {1} group by {0}".format(self.key, self.tableName)
        result = self.doQuery(query)
        self.vals = result
        return result


class RelTable(DBAbstr):
    leftVals = DictOfSets()
    rightVals = DictOfSets()

    def __init__(self, tableName, setA, translKeyA, setB, translKeyB):
        self.tableName = tableName
        self.setA = setA
        self.translKeyA = translKeyA
        self.setB = setB
        self.translKeyB = translKeyB

    def getLeftVals(self):
        if self.leftVals.data:
            return self.leftVals.data
        query  = "select a." + self.setA.key + ", c." + self.setB.key + " "
        query += "from " + self.setA.tableName + " as a "
        query += "left join " + self.tableName + " as b on b." + self.translKeyA + " = a." + self.setA.key + " "
        query += "left join " + self.setB.tableName + " as c on c." + self.setB.key + " = b." + self.translKeyB + " "
        query += "group by a." + self.setA.key + ", c." + self.setB.key
        result = self.doQuery(query)
        self.leftVals.addData(result)
        return result

    def getRightVals(self):
        if self.rightVals.data:
            return self.rightVals.data
        query  = "select a." + self.setB.key + ", c." + self.setA.key + " "
        query += "from " + self.setB.tableName + " as a "
        query += "left join " + self.tableName + " as b on b." + self.translKeyB + " = a." + self.setB.key + " "
        query += "left join " + self.setA.tableName + " as c on c." + self.setA.key + " = b." + self.translKeyA + " "
        query += "group by a." + self.setA.key + ", c." + self.setB.key
        result = self.doQuery(query)
        self.rightVals.addData(result)
        return result





class Relation:
    vals = DictOfSets()     # x  ->  f(x) = y
    invVals = DictOfSets()  # y  ->  f^-1(y) = x

    def __init__(self, setTableX, setTableY, relTable):
        self.setTableX = setTableX
        self.setTableY = setTableY
        self.relTable = relTable

    def checkTransitive(self):
        pass

    def checkBijective(self):
        """ One-to-one and onto """
        return self.checkSurjective and self.checkInjective

    def checkSurjective(self):
        """ Onto: For any y in Y, there is a x in X: y = f(x) """
        invValsData = self.getInvValsData()
        for y in invValsData:
            if len(invValsData[y]) == 0:
                return False
        return True

    def checkInjective(self):
        """  Ont-to-One: For any a, b in X: f(a) = f(b) => a = b """
        invValsData = self.getInvValsData()
        for y in invValsData:
            if len(invValsData[y]) > 1:
                return False
        return True
            

    def compose(self, rel2):
        pass


    def getValsData(self):
        if  self.vals.data:
            return self.vals.data
        xdata = self.setTableX.getVals()
        self.vals.addKeys(xdata)
        reldata = self.relTable.getLeftVals()
        self.vals.addVals(reldata)
        return self.vals.data


    def getInvValsData(self):
        if self.vals.data:
            return self.vals.data
        ydata = self.setTableY.getVals()
        self.invVals.addKeys(ydata)
        reldata = self.relTable.getRightVals()
        self.invVals.addVals(reldata)
        return self.invVals.data




if __name__ == "__main__":
    pnstMappen = SetTable('gkd_chemie.LIMNO_PNST_MAPPE', 'MAPPE_NR')
    messnetze = SetTable('gkd_chemie.LIMNO_SL_PNST_MAPPE_TYP', 'TYP_NR')
    pnstMappePnst = RelTable('gkd_chemie.LIMNO_ZT_PNST_MAPPE_PNST', pnstMappen, 'MAPPE_NR', messnetze, 'MN_NR')
    mappenMessnetzRel = Relation(pnstMappen, messnetze, pnstMappePnst)
    isInj = mappenMessnetzRel.checkInjective()
    isSur = mappenMessnetzRel.checkSurjective()
    print isInj
    print isSur

\end{lstlisting}



\subsection{Distributed cache: redis}
