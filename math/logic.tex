\section{Logic}

\subsection{Algebra}

$\land$ and $\lor$ are distributive: 

$$ (A \land B) \lor C \equiv (A \lor C) \land (B \lor C)$$

$$ (A \lor B) \land C \equiv (A \land C) \lor (B \land C) $$

This is easiest seen by drawing a Venn-diagram. 

\subsection{Quantifiers}

$$ \thereis ! x: Q(x) \equiv ( \thereis x: Q(x) ) \land ( \forall x,y: Q(x) \land Q(y) \then y = x ) $$

\subsection{Inference}

Simple if statements: 

$$ A \then B \equiv \lnot A \lor B $$

$$ \lnot (A \then B) \equiv A \land \lnot B$$


Applying this to function-statements yields: 

$$ \lnot ( \forall x: A(x) \then B(x) ) \equiv \thereis x: A(x) \land B(x)$$ 

If-and-only-if statements: 

$$ A \iff B \equiv (A \then B) \land (B \then A) \equiv (A \land B) \lor (\lnot A \land \lnot B)$$

$$ \lnot (A \iff B) \equiv (A \land \lnot B) \lor (B \land \lnot A) \equiv (\lnot A \then B) \land (A \then \lnot B) \equiv A \iff \lnot B$$

Applying this to function-statements yields: 

$$ \lnot (\forall x : A(x) \iff B(x)) \equiv \thereis x : ( \lnot A(x) \then B(x) ) \land ( A(x) \then \lnot B(x) ) $$


\subsection{Consistency and well-definedness}

Whenever you receive a set of axioms (like \emph{imageine there was a complex number i}), you should only accept working with them when they don't lead to any logical inconsistencies. However, it is hard to prove that a set of axioms doesn't lead to inconsistencies. Usually, you start with a really basic set (ZFC axioms) and prove that you can construct structures (like for example Dedekind cuts) that have as properties the new axioms you want to establish. Then your axioms are consistent as long as ZFC is consistent. 

A concept is well defined when it is a function, that is, when for every input-data there is always at most one output-data. 


\subsection{Experimental and protocol design}

This section deals with logic-puzzles. 

\paragraph{Multistep processes}
At every step of your experiment, utilize the full spectrum of your measuring device. This way, each measurement contains the maximal information content. For example, in the "twelve coins" problem, at every one of your three steps the scale should be able to tip left, right or not at all. 
