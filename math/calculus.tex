\section{Calculus}

\subsection{Hyperreals}

For the biggest part, we're going to deal with Nelson-style nonstandard-analysis. 

List of external properties:
\begin{itemize}
    \item std, nstd
    \item limt, nlimt
    \item inftsm, inft
    \item nearly cont
\end{itemize}

A statement using any external properties will be denoted as $\ext{A}$, one that \emph{might} use external properties as $\pext{A}$.

We will use the following axioms:
\begin{enumerate}
    \item $0:std$
    \item $\forall n \in \naturals: n:std \then (n+1):std $
    \item $\thereis n \in \naturals: n:nstd $
    \item External induction: Induction  over \std{n} about \pext{A}: \\ 
    $ [ \pext{A}(0) \land \forall \std{n} \in \naturals: \pext{A}(n) \then \pext{A}(n+1) ] \then \forall \std{n} \in \naturals: \pext{A}(n)$
    \item Internal induction: Induction over \pnstd{n} about A: \\
    $ [A(0) \land \forall \pnstd{n} \in \naturals: A(n) \then A(n+1)] \then \forall n \in \naturals: A(n) $
\end{enumerate}

The rationale over the two induction-axioms is simple. Ordinary induction is about $A$ over \std{n}. 
External induction is about \pext{A} over \std{n}. This makes sure that statements about external stuff only apply to finite $n$, not to infinite ones. 
Internal induction is about $A$ over \pnstd{n}. This makes sure that when we talk about potentially infinite $n$'s, we only apply internal statements.

In other words: these two inductions ensure that we \textbf{never apply external statements to external numbers}. 

Doing so would lead to logical incosistencies. That's why there is no "fully external" induction.

However, note that axiom 2 is actually a case of "fully external" induction.


\begin{theorem}
    $ \forall n \in \naturals: n:nst \then (n+1):nst $ \label{addingNonstds}
\end{theorem}
\begin{proof}
    \subprf{Suppose $n:nst$.}{$(n+1):nst$}{
        \subprf{By contradiction. Suppose $(n+1):std$.}{this leads to a contradiction.}{
            $(n+1):std \then n:std$.  \\
            This contradicts the premise that $n:nst$.
        }
    }
\end{proof}

If you don't believe the argument in the previous proof, consider this: 
\begin{proof}
    \subprf{
    Suppose all of the following: \\
    $ [\forall n: Q(n) \then Q(n+1)] \then \forall n: Q(n) $ \\
    $ [\forall n: Q(n) \then Q(n+1)] $ \\
    This leads to $ \forall n: Q(n) $. \\ 
    }{$ \forall n: Q(n+1) \then Q(n) $}{
        \subprf{Let $n = n_0$ and suppose $Q(n_0+1)$.}{$Q(n_0)$}{
            Since $ \forall n: Q(n) $ holds, it must be true that $Q(n_0)$.
        }
    }
\end{proof}


We can use theorem \ref{addingNonstds} to prove the following: 

\begin{theorem}
    $\forall n, m \in \naturals: n:std \land m:nstd \then (n+m):nst$
\end{theorem}
\begin{proof}
    \subprf{Let $m=m_0:nst$.}{$\forall n \in \naturals: n:std  \then (n+m_0):nst$}{
        By induction on $n$. 
        \\
        \subprf{Base case. Let $n=0$.}{$(0+m_0):nst$}{
            $0+m_0=m_0$ \\
            $m_0:nst$
        }
        \\
        \subprf{Induction step.}{$[(n+m_0):nst] \then [(n+1+m_0):nst]$}{
            Just apply theorem \ref{addingNonstds} to $n=(n+m_0)$.
        }
    }
\end{proof}

It is notable that you can never reach a standard number when adding nonstandard numbers.
\begin{theorem}
    $ \forall n,m \in \naturals: n,m:nst \then (n+m):nst $
\end{theorem}
\begin{proof}
    We proceed by proving the equivalent $ (n+m):std \then (n:std \lor m:std) $ 
    \subprf{Suppose $(n+m):std$}{$(n:std \lor m:std)$}{
        \subprf{Without loss of generality, suppose $n:nst$}{$m:std$}{
            \subprf{By contradiction. Suppose $m:nst$}{this leads to a contradiction}{
                We have already assumed that  $(n+m):std$. \\
                Now, however, we also assume that $n,m:nst$. \\
                Using theorem \ref{addingNonstds} however, we see that when $n,m:nst$, then it must be that $(n+m):nst$.
            }
        }
    }
\end{proof}


\subsection{Limits}

\subsection{Sequences and series}

\subsubsection{Tailor}
\subsubsection{Fourier}
\subsubsection{Laplace}

\subsection{Euler's formula}
Proof: Consider the function $f(t)=e^{-it}(cost+isint)$ for $t \in \reals$. By the quotient rule
$$ f'(t) = e^{-it} (i\ cos(t) - \sin(t)) -ie^{-it} (\cos(t) + i \sin(t)) = 0 $$
identically for all $t \in \reals$. Hence, $f$ is constant everywhere. Since $f(0)=1$, it follows that $f(t)=1$ identically. Therefore, $e^{it}=cost+isint$ for all $t \in \reals$, as claimed.


\subsection{Integration}

The infinitessimal view of calculus makes it quite easy to prove theorems. Consider this illustration of how definite integrals work.

\begin{equation}
    \begin{aligned}
        \int_a^b \frac{dF}{dx}(x) dx &= \int_a^b \frac{F(x + dx) - F(x)}{dx} dx \\
        \int_a^b F(x + dx) - \int_a^b F(x) &= F(b + dx) - F(a)
    \end{aligned}
\end{equation}


\subsubsection{Integration strategies}

\paragraph{u-substitution}

\paragraph{Integration by parts} is the last trick up our sleave when all other strategies haven't helped. Consider the integral 

$$ \int x e^x \diff{x} $$

We can rewrite this integral as 

$$ \int u \diff{v} $$

where $u = x$ and $\diff{v} = e^x \diff{x}$. In general, you always want to pick $u$ in such a way that $u$ gets simpler after being differentiated.
$x$ does get a lot simpler after differentiation, whereas $e^x$ doesn't, so the choice is clear. 

We then use the following: 

$$ \int u \diff{v} = uv - \int v \diff{u} $$\footnote{The proof goes like this: Starting with the product rule of differentiation: $(ab)' = a'b + ba'$ we get $(ab)' - a'b = ba'$ and  $ab - \int a'b \diff{x} = \int ba' \diff{x} $}

Since we chose $u = x$ we have $\diff{u} = \diff{x}$, and from $\diff{v} = e^x \diff{x}$ we get $v = e^x$. This yields us: 

$$ xe^x - \int e^x \diff{x} $$
$$ e^x ( x - 1) $$

as the solution. 

\subsection{Vector calculus}

Integration over a vector, integration along a vector, integration along a surface. 

You can find a nice introduction (mostly in the second part of) this pdf: \inlinecode{http://www.maths.gla.ac.uk/~cc/2A/2A_notes/2A_chap4.pdf} and here \inlinecode{http://geocalc.clas.asu.edu/pdf-preAdobe8/SIMP_CAL.pdf}
