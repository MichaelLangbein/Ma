\section{Tensorflow}

\subsection{Low-level API}

Working with tf always entails a two-step process. First, define a graph, and second, execute it. 

\begin{lstlisting}[language=python]
import tensorflow as tf

# Step 1: define a graph
a = tf.Variable(initial_value=2)
b = tf.Variable(initial_value=3)
c = a * b

# Step 2: execute graph
with tf.Session() as sess:
    sess.run(tf.global_variables_initializer())
    result = sess.run(c)
    print(result)

\end{lstlisting}

\emph{Running} a tensor means evaluating it - in the case of \inlinecode{c}, that means calculating \inlinecode{a * b}.
There are two things that can be run by a session: tensors (to be evaluated) and operations (to be executed). 

Optimizers are a very common example of operations, so we show a very simple example of their usage.

\begin{lstlisting}[language=python]
import tensorflow as tf
import numpy as np

# Step 0: Data
a = 0.2
b = 5
xs_training = np.random.random(100) * 100
ys_training = a*xs_training + b + np.random.random(100)
xs_validation = np.random.random(100) * 100
ys_validation = a*xs_validation + b + np.random.random(100)

# Step 1: define model
a_tf = tf.Variable(0.3421)
b_tf = tf.Variable(0.41)
prediction_tf = a_tf * xs_training + b_tf
loss_tf = tf.reduce_sum((ys_training - prediction_tf)**2)
optimizer = tf.train.GradientDescentOptimizer(0.1)
training_operation = optimizer.minimize(loss_tf, var_list=[a_tf, b_tf])

# Step 2: execute model
with tf.Session() as sess:
    sess.run(tf.global_variables_initializer())
    for i in range(10):
        results = sess.run([training_operation, a_tf, b_tf])
        print(results)


\end{lstlisting}