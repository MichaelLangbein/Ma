\section{Jekyll}
Jekyll is a ruby/markdown based static site builder.


\subsection{Setup}

We want to use github's user-account hosting with our jekyll site. For this reason, our site's project's name must equal our github-useraccount: 

\begin{lstlisting}
gem install jekyll bundler
jekyll new MichaelLangbein.gitbub.io
cd MichaelLangbein.gitbub.io
git add --all
git commit -m "Let's go!"
git push
\end{lstlisting}

Now we have a github-repository named \verb MichaelLangbein.github.io  which github will recognize and automatically build and host. 



\subsection{Blogging}
\paragraph{To start blogging} just add files to your \verb _posts  directory.

\paragraph{To add sites other than block entries} it is best to ...

\paragraph{To add a navigation menu} you can ...



\subsection{Layout}
\paragraph{Themes} define the basic layout of your site. If you don't like all the details of your theme, you can always overwrite them by customizing your layout by adding a \verb _layouts  folder to your project. 
Chosing a theme is as simple as editing your \verb _config.yml  file: 
\begin{lstlisting}
theme: jekyll-theme-cayman
\end{lstlisting}

\paragraph{Customizing layout} You use layouts when you want to have full control over all html. All the stuff you put in your \verb _layouts  folder automatically overwrites whichever theme you have chosen. There is a lot of good information \hyperlink{https://learn.cloudcannon.com/jekyll/introduction-to-jekyll-layouts/}{here}.