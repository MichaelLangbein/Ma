
\section{React}

\subsection{Setup}
We follow the instructions from here: https://www.codecademy.com/articles/how-to-create-a-react-app
\begin{lstlisting}
sudo npm install -g create-react-app
create-react-app myapp
cd myapp

\end{lstlisting}

\subsection{JSX}

React uses an extension of javascript, called JSX, that first needs to be compiled into regular javascript. JSX enables you to put pure html into your sourcecode, like this: 

\begin{lstlisting}
var myDiv = (
  <div>
    <h1>Hello world</h1>
  </div>
);
\end{lstlisting}

Any such element can then be rendered:
\begin{lstlisting}
import React from 'react';
import ReactDOM from 'react-dom';

ReactDOM.render(<h1>Hello world</h1>, 
               document.getElementById('app'));
\end{lstlisting}

You can even put js again inside jsx! Just make sure to include any js-expression inside \emph{curly} braces. 
\begin{lstlisting}
ReactDOM.render(
	<h1>{2 + 3}</h1>,
  document.getElementById("app")
);
\end{lstlisting}

Also works perfectly fine with functions.
\begin{lstlisting}
import React from 'react';
import ReactDOM from 'react-dom';

function makeDoggy(e) {
  e.target.setAttribute('src', 'https://s3.amazonaws.com/puppy.jpeg');
  e.target.setAttribute('alt', 'doggy');
}

const kitty = (
	<img 
		src="https://s3.amazonaws.com/codecademy-content/courses/React/react_photo-kitty.jpg" 
        onClick={makeDoggy}
    />
);

ReactDOM.render(kitty, document.getElementById("app"));
\end{lstlisting}

\paragraph{Common pitfalls with JSX} are listed here: 
\begin{itemize}
    \item Html-elements mustn't use the keyword \verb class  . Use \verb className  instead. 
    \item In self-closing tags, you \emph{must} include the closing slash. This is correct \inlinecode{<br/>} while this isn't: \inlinecode{<br>}.
    \item in HTML, event listener names are written in all lowercase, such as onclick or onmouseover. In JSX, event listener names are written in camelCase, such as onClick or onMouseOver
    \item No if-statements are allowed inside JSX-{}. Use three-way-comparison operators instead. 
\end{itemize}


\subsection{Components}
Components allow you to define custom html-elements. 
\begin{lstlisting}
import React from 'react'; // react-core
import ReactDOM from 'react-dom'; // for interacting with dom

class MyComponentClass extends React.Component {
  render() {
    return <h1>Hello world</h1>;
  }
}

ReactDOM.render(
	<MyComponentClass />, 
	document.getElementById('app')
);
\end{lstlisting}
Note that the \inlinecode{extends} class-syntax is just ES6 syntactic sugar. Under the hood, javascript is still using prototypes.

Very often, components contain event-handlers:
\begin{lstlisting}
class Example extends React.Component {
  handleEvent() {
    alert(`I am an event handler.
      If you see this message,
      then I have been called.`);
  }

  render() {
    return (
      <h1 onClick={this.handleEvent}>
        Hello world
      </h1>
    );
  }
}
\end{lstlisting}

\subsection{Props and state}
Every component has a field named \inlinecode{props}.
\begin{lstlisting}
import React from 'react';
import ReactDOM from 'react-dom';

class PropsDisplayer extends React.Component {
  render() {
  	const stringProps = JSON.stringify(this.props);

    return (
      <div>
        <h1>CHECK OUT MY PROPS OBJECT</h1>
        <h2>{stringProps}</h2>
      </div>
    );
  }
}


ReactDOM.render(<PropsDisplayer name="Frarthur" town="Flundon" age={2} haunted={false} />, document.getElementById("app"));
\end{lstlisting}

You can, and often will, pass functions, like event-handlers, as props. 


State is implemented by adding a variable named \inlinecode{state} to the Component-class. You can then alter the state by calling \inlinecode{setState}, which as a side-effect calls \inlinecode{render} again, such that your state-change is immediately reflected in the layout. 

\begin{lstlisting}
import React, { Component } from 'react';
import logo from './logo.svg';
import './App.css';

const list = [
  {
    title: "React",
    url: "https://facebook.github.io/react/",
    author: "me",
    num_comments: 3,
    points: 4,
    objectID: 0,
  },
  {
    title: "Redux",
    url: "https://github.com/reactjs/redux",
    num_comments: 2,
    author: "someone else",
    points: 5,
    objectID: 1,
  }
];

class App extends Component {

  constructor(props) {
    super(props); // allways call super(props)
    this.state = { // must be called state
      list: list
    };
    // props and state are very similar; but state is private.
    this.onDismiss = this.onDismiss.bind(this); // all your custom methods must first be registered (except if they are lifecycle hooks)
  }

  onDismiss(objId) {
    const isNotId = item => item.objectID !== objId;
    const shorterList = this.state.list.filter(isNotId);
    this.setState({ list: shorterList }); // automatically calls render() again
  }

  render() {

    var myName = "Michael";

    return (
      <div className="App">
        <div className="App-header">
          <img src={logo} className="App-logo" alt="logo" />
          <h2>Hello there, {myName}!</h2>
        </div>

        {this.state.list.map( item => {
          return (
            <div key={item.objectID}>
              <span><a href={item.url}>{item.title}</a></span>
              <span>{item.author}</span>
              <span>{item.num_comments}</span>
              <span>{item.points}</span>
              <span><button onClick={() => {this.onDismiss(item.objectID)}}>Dismiss</button></span>
            </div>
          );
        })}

      </div>
    );
  }

}

export default App;

\end{lstlisting}