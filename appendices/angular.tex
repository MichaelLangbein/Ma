\section{Angular}

\begin{itemize}
    \item Module: lists all components (\inlinecode{declarations}) and all dependencies (\inlinecode{imports})
    \item Component: Defines a custom html-Tag
    \item Service: accesses a server
\end{itemize}

\subsection{Components}

A component consists of a html-template and a ts-class. The template can access the classes properties and methods through directives.
These directives are special syntax like \inlinecode{\{\}} or \inlinecode{*ngIf} that contain a string of ts-code that can be evaluated in the context of the class.

\begin{itemize}
    \item ts-file: has a method for every lifecycle-hook and a property for every bindable data.
    \item html-file:
            \begin{itemize}
                \item interpolation: curly braces can read a classes properties. \inlinecode{<li>\{\{hero.name | uppercase\}\}</li>}
                \item input-binding: round braces capture input-events. \inlinecode{(click)="onSelect(hero)"} Captures an input and directs to a method
                \item value-binding: square braces write to a classes properties. \inlinecode{[hero]="chosenHero"} assigns an object to a property. 
                \item two-way-binding: for reading \emph{and} writing properties. \inlinecode{[(ngModel)]="hero.name"}, requires \inlinecode{FormsModule}
                \item loops: \inlinecode{*ngFor="let hero of heroes"}
                \item conditionals: \inlinecode{*ngIf="!selected"}
            \end{itemize}
            Together, these might make a code section like this: \inlinecode{<li *ngFor="let hero of heroes" (click)="onSelect(hero)">\{\{hero.id\}\} -- \{\{hero.name\}\}</li>}
\end{itemize}

\subsubsection{Template directives: details}

\paragraph{*ngFor}

\paragraph{*ngIf} 

\subsubsection{Subcomponents}

Subcomponents can obtain the values of their properties from supercomponents. 

A component-class can have properties annotated with \inlinecode{@Input}. That means that the value of this property will be given through the components html-tag. 
For example, the component \inlinecode{Heroes} might have this tag in its template:

\begin{lstlisting}[language=html]
    <app-hero-detail [hero]="chosenHero"></app-hero-detail>
\end{lstlisting}

This \inlinecode{chosenHero} would be bound to the \inlinecode{hero} property in the \inlinecode{HeroDetail} component.

\begin{lstlisting}[language=java]
export class HeroDetailComponent implements OnInit {

  @Input()
  hero: Hero;

  constructor() {}
  ngOnInit() {}
}
\end{lstlisting}


\subsection{Routing}
Routing is usually done in a separate module. It consist of two parts: 
\begin{itemize}
    \item The route definition, consisting of 
    \item The html-links
\end{itemize}
As a side effect, when you decide to use routes, not all components need to be subcomponents of \inlinecode{app-component} anymore.


\subsection{Modules}
You create modules when you want to create some functionality that might be shared via npm. 
Modules are always imported into angular in the form of a class. To import your module into a site, 
\begin{itemize}
    \item download the module with npm
    \item \inlinecode{ng generate module <yourmodule>}
    \item add the module to the app-modules \inlinecode{import}s
    \item in \inlinecode{<yourmodule>.module.ts}, 
        \begin{itemize}
            \item create a class \inlinecode{<Yourmodule>Module}
            \item configure that class with \inlinecode{@ngModule}
            \item export the class
        \end{itemize}
\end{itemize}


\subsection{Services}
Services are data providers. They do not have to call REST-interfaces. They might call a database or a file, or they might be used as a messaging service between components. But if they do happen to call a REST-service, they typically return an observable.


\subsection{General concepts}

As a singlethreaded, event-based environment, javascript has a bunch of event-based structures built in.

\paragraph{Observables} are objects that maintain a list of subscribers. The subscribers must implement the subscriber-interface, which means that they must have methods \inlinecode{next, completed, error}.
The "notify-all-subscribers" part is handled by an abstract base class from \inlinecode{RxJS}. To create an observable, you only need to pass in a function that tells the observable how to  get its data.


\begin{lstlisting}[language=java]
let observable = Rx.Observable.create((observer) => {
    for (var i = 0; i<10; i++) {
        let data = someRestRequest("myservice.com/data/" + i);
        observer.onNext(data);
    }
    observer.onFinished();
});
    
let observer = {
    onNext: (nextVal) => console.log(nextVal),
    onCompleted: () => console.log("completed"),
    onError: (error) => console.log(error)
};

observable.subscribe(observer);
\end{lstlisting}
