\section{Remote sensing}


\subsection{Electromagnetic radiation}
Satellites all work the same way: they are flying machines that carry a camera. The camera has one or more sensors, each of which can observe some part of the electromagnetic spectrum.
Radar satellites have their own 'light' source: they actively send out radio-radiation and collect the reflection.

\begin{itemize}
    \item Ultra-violet. Breaks molecular bonds. 
    \item Optical. Electrons get in an excited state, and send out a photon again when they fall back down.
    \item NIR: heat.
    \item Radar: not hindered by clouds. Good for flood-detection. \begin{itemize}
        \item Microwave: good for atmospheric observation. Makes molecules vibrate.
    \end{itemize}
\end{itemize}

EM-radiation has not only a frequency - it is also characterized by its \emph{phase}.




\subsection{Orbital periods and acquisition}
Satellites usually have an orbital period of 1 - 2 hours. They revisit the same spot on earth every 2-3 weeks.
(This example data is taken from landsat: an orbit takes 99 minutes, and it visits the same spot every 16 days.
Note that landsat always stays on the sunny side of earth - most visible-light satellites do this.)
Commonly, satellite series have offset periods so one machine of the same series
visits the same spot every nth of a full orbital period.
However, some satellites can narrow or broaden their sensors, so even if
a satellite visits the same spot again, it might not make the same snapshot again.
To handle the high demand and low availability of satellite images, acquisition plans are being made.
Usually, these take into account seasonal effects,
water vapor, and also a ranking of requests from different scientific institutions.


Note that not all satellites can change their focus. Sentinel-5P, for example, does \emph{sweep broom staring}, i.e. always looks down at exactly the same angle and swath width.
Consequently, there are no acquisition plans for Sentinel-5P.


Important orbits:
\begin{itemize}
    \item Lower earth orbit: 160-1000 km above sea level. ISS is at 600 km, OCO2 at 700 km. ~15.000 km/h for OCO. \begin{itemize}
        \item A-track: a.k.a. the afternoon track. Passes equator from south at 14:00.
    \end{itemize}
    \item Geo-stationary: 30.000 km above sea level. For communication- and relay-satellites. ~ 10.000 km/h.
    \item Moon: 300.000 km
\end{itemize}




\subsection{Important satellites}


\begin{table}
    \centering
    \begin{tabular}{llllllll}
        \toprule
        Satellite   & Owner & Bands & Data available at                    & Started    & Orbit                                                         & Resolution & Description                                               \\
        \midrule
        Landsat     & NASA  &       & USGS Earth explorer / AWS-S2 as COGs &            &                                                               &            & US optical, very long running                             \\
        Modis       & NASA  &       &                                      &            &                                                               &            & Optical (but very coarse) / Temperature, water-vapor      \\
        Copernicus  & ESA   &       &                                      &            &                                                               &            &                                                           \\
        Sentinel 1  & ESA   &       & Sentinel-hub EO-browser              & 03.04.2014 & Near-polar, sun-sync, 1.5h, 6-day world coverage (using both) & 5m * 5m    & Radar                                                     \\
        Sentinel 2  & ESA   &       & Sentinel-hub EO-browser              &            &                                                               &            & Optical                                                   \\
        Sentinel 3  & ESA   &       & Sentinel-hub EO-browser              &            &                                                               &            & Sea surface topography, temperature (also fire detection) \\
        Sentinel 5p & ESA   &       & Sentinel-hub EO-browser              &            &                                                               & 6km * 4km  & Atmospheric gasses                                        \\
        TerraSarX   & DLR   &       &                                      & 15.06.2007 &                                                               & 1m/3m/16m  & Radar, very high resolution, commercial/scientific coop   \\
        RapidEye    &       &       &                                      &            &                                                               &            &                                                           \\
        Ikonos      &       &       &                                      &            &                                                               &            &                                                           \\
        \bottomrule
    \end{tabular}
\end{table}

\paragraph{Landsat} program is the longest-running enterprise for acquisition of satellite imagery of Earth.
On July 23, 1972 the Earth Resources Technology Satellite was launched. This was eventually renamed to Landsat. 
The most recent, Landsat 8, was launched on February 11, 2013. The instruments on the Landsat satellites have acquired millions of images.
The images, archived in the United States and at Landsat receiving stations around the world,
 can be viewed through the U.S. Geological Survey (USGS) 'EarthExplorer' website.
 Landsat 7 data has eight spectral bands with spatial resolutions ranging from 15 to 60 meters;
 the temporal resolution is 16 days.[2] Landsat images are usually divided into scenes for easy downloading.
 Each Landsat scene is about 115 miles long and 115 miles wide (or 100 nautical miles long and 100 nautical miles wide, or 185 kilometers long and 185 kilometers wide).

\paragraph{Modis} ...

\paragraph{Copernicus} ...

\paragraph{ENVISAT} was ESA's environmental satellite. First satellite to measure CO2. Huge: large as a bus. 
Launched 2002, scheduled until 2007, eventually lost contact 2012. Didn't manage to go to graveyard orbit - still hangs around at 800km.
Passes two objects a year > 10kg within 200m. Danger of causing Kessler syndrome.
Since replaced by the Sentinel-series.

\paragraph{Sentinel} ESA is currently developing seven missions under the Sentinel programme.
The Sentinel missions include radar and super-spectral imaging for land, ocean and atmospheric monitoring.
Each Sentinel mission is based on a constellation of two satellites to fulfill and revisit the coverage requirements
for each mission, providing robust datasets for all Copernicus services.

\paragraph{Sentinel-1} Began 2014. Always in pairs. 700km. 12 day repeat cycle. 
C-Band only. Huge data. Single scene is 4 GB in size.
Usually operates in Wide-Swath mode (5*20m, 250km swath), but can be tasked to focus on regions in strip-map mode (5*5m, 80km swath).


\paragraph{OCO-2 and OCO-3} are NASA's CO2 monitors. OCO-2 is in A-track at 700km. Launched 2014, supposed to last for a few more years.
OCO-3 is docked to the japanese Lab on ISS (ISS itself intended to last until ~2032).


\subsection{Service providers}

\paragraph{Amazon S3} keeps a lot of satellite data as COGs. \begin{itemize}
    \item Sentinel 1 from \href{https://registry.opendata.aws/sentinel-1/}{here}
    \item Sentinel 2 from \href{https://sentinel-cogs.s3.us-west-2.amazonaws.com/sentinel-s2-l2a-cogs/2020/S2A_36QWD_20200701_0_L2A/TCI.tif}{here}
    \item Landsat from \href{https://landsat-pds.s3.amazonaws.com/c1/L8/139/045/LC08_L1TP_139045_20170304_20170316_01_T1/LC08_L1TP_139045_20170304_20170316_01_T1}{here}
\end{itemize}


\paragraph{EODC}: \href{https://www.eodc.eu/}{Earth Observation Data Center for Water Resources Monitoring}

\paragraph{Eurac}: \href{http://www.eurac.edu}{Eurac} is a private research company ...

\paragraph{Google Earth Engine} provides ...

\paragraph{NASA's ECS} (Earth observation center Core System) is a vast catalogue of ...




\subsection{Image preprocessing}
\begin{itemize}
    \item Geometric distortions
    \item Georeferencing
\end{itemize}


\subsection{Radar}
\begin{itemize}
    \item Earth does not emit any radar (in fact, only lightning and some pulsars do emit radar naturally). So, radar must be an active sensor. Advantage: works at night.
    \item The higher the wavelength, the better the penetration power: L&P bands go through foliage and into the topsoil.
    \item The lower the wavelength, the better the resolution.
    \item Two data-products: amplitude and phase.
    \item Side-facing. This way water can specularly reflect radar and appear perfectly black in image.
\end{itemize}



\subsection{Image segmentation}
\subsubsection{Maximum-likelyhood classifier}
\subsubsection{PCA}
\subsubsection{U-Net}