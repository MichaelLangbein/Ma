\section{Remote sensing}


\subsection{Electromagnetic radiation}
Satellites all work the same way: they are flying machines that carry a camera. The camera has one or more sensors, each of which can observe some part of the electromagnetic spectrum.
Radar satellites have their own 'light' source: they actively send out radio-radiation and collect the reflection.

\begin{itemize}
    \item Optical
    \item Radar: not hindered by clouds. Good for flood-detection
    \item Microwave: good for atmospheric observation
\end{itemize}





\subsection{Orbital periods and acquisition}
Satellites usually have an orbital period of 1 - 2 hours. They revisit the same spot on earth every 2-3 weeks.
(This example data is taken from landsat: an orbit takes 99 minutes, and it visits the same spot every 16 days.
Note that landsat always stays on the sunny side of earth - most visible-light satellites do this.)
Commonly, satellite series have offset periods so one machine of the same series
visits the same spot every nth of a full orbital period.
However, some satellites can narrow or broaden their sensors, so even if
a satellite visits the same spot again, it might not make the same snapshot again.
To handle the high demand and low availability of satellite images, acquisition plans are being made.
Usually, these take into account seasonal effects,
water vapor, and also a ranking of requests from different scientific institutions.








\subsection{Important satellites and service providers}


\begin{table}
    \centering
    \begin{tabular}{llllllll}
        \toprule
        Satellite   & Owner & Bands & Data available at                    & Started    & Orbit                                                         & Resolution & Description                                               \\
        \midrule
        Landsat     & NASA  &       & USGS Earth explorer / AWS-S2 as COGs &            &                                                               &            & US optical, very long running                             \\
        Modis       & NASA  &       &                                      &            &                                                               &            & Optical (but very coarse) / Temperature, water-vapor      \\
        Copernicus  & ESA   &       &                                      &            &                                                               &            &                                                           \\
        Sentinel 1  & ESA   &       & Sentinel-hub EO-browser              & 03.04.2014 & Near-polar, sun-sync, 1.5h, 6-day world coverage (using both) & 5m * 5m    & Radar                                                     \\
        Sentinel 2  & ESA   &       & Sentinel-hub EO-browser              &            &                                                               &            & Optical                                                   \\
        Sentinel 3  & ESA   &       & Sentinel-hub EO-browser              &            &                                                               &            & Sea surface topography, temperature (also fire detection) \\
        Sentinel 5p & ESA   &       & Sentinel-hub EO-browser              &            &                                                               & 6km * 4km  & Atmospheric gasses                                        \\
        TerraSarX   & DLR   &       &                                      & 15.06.2007 &                                                               & 1m/3m/16m  & Radar, very high resolution, commercial/scientific coop   \\
        RapidEye    &       &       &                                      &            &                                                               &            &                                                           \\
        Ikonos      &       &       &                                      &            &                                                               &            &                                                           \\
        \bottomrule
    \end{tabular}
\end{table}

\paragraph{Landsat} program is the longest-running enterprise for acquisition of satellite imagery of Earth. On July 23, 1972 the Earth Resources Technology Satellite was launched. This was eventually renamed to Landsat.[1] The most recent, Landsat 8, was launched on February 11, 2013. The instruments on the Landsat satellites have acquired millions of images. The images, archived in the United States and at Landsat receiving stations around the world, are a unique resource for global change research and applications in agriculture, cartography, geology, forestry, regional planning, surveillance and education, and can be viewed through the U.S. Geological Survey (USGS) 'EarthExplorer' website. Landsat 7 data has eight spectral bands with spatial resolutions ranging from 15 to 60 meters; the temporal resolution is 16 days.[2] Landsat images are usually divided into scenes for easy downloading. Each Landsat scene is about 115 miles long and 115 miles wide (or 100 nautical miles long and 100 nautical miles wide, or 185 kilometers long and 185 kilometers wide).

\paragraph{Modis} ...

\paragraph{Copernicus} ...

\paragraph{Sentinel} ESA is currently developing seven missions under the Sentinel programme. The Sentinel missions include radar and super-spectral imaging for land, ocean and atmospheric monitoring. Each Sentinel mission is based on a constellation of two satellites to fulfill and revisit the coverage requirements for each mission, providing robust datasets for all Copernicus services.

\paragraph{CHIRPS} is ...

\paragraph{EODC}: \href{https://www.eodc.eu/}{Earth Observation Data Center for Water Resources Monitoring}

\paragraph{Eurac}: \href{http://www.eurac.edu}{Eurac} is a private research company ...

\paragraph{Google Earth Engine} provides ...

\paragraph{NASA's ECS} (Earth observation center Core System) is a vast catalogue of ...










\subsection{Image preprocessing}
\subsubsection{}
\subsection{Radar}
\subsection{Image segmentation}
\subsubsection{Maximum-likelyhood classifier}
\subsubsection{PCA}
\subsubsection{U-Net}