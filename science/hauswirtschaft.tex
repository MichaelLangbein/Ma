\section{Elektronik}

\subsection{Mathe}
\paragraph{Ohmsches Gesetz} ...
\paragraph{Kirchoffsche Regeln} ...

\subsection{Bauteile}
\paragraph{Multimeter}\dots
\paragraph{Kabelfinder} \dots
\paragraph{Netzteil} aka. PSU: wandelt Wecheslstrom in Gleichstrom um. Eingangsstrom kann beliebig polarisiert sein (i.e. Stecker kann auf 2 verschiedene Weisen in Steckdose stecken), Ausgangsstrom-Polarisierung ist immer die gleiche, unabhängig von der Drehung des Eingangssteckers.

\subsection{Architektur}

\paragraph{Hausstrom}...
\begin{itemize}
    \item 230 V Wechelsspannung
    \item Braun: Phase aka. Außenleiter (L)
    \item Blau: Neutralleiter (N)
    \item Grün & Gelb: Erdung. Erdungskabel immer etwas länger als Phase & Neutral lassen, damit dieses Kabel als letztes abreißt. Verbunden mit Zentralsicherung.
\end{itemize}


\paragraph{Verbauung}...
\begin{itemize}
    \item Wände \begin{itemize}
        \item Kabel laufen vertical durch Stromstecker zu Boden oder zu Decke
        \item Im Bad zusätzlich noch 1.2m Sicherheitsabstand zu Dusche, Toillette, Waschbecken.
    \end{itemize}
    \item Decke \begin{itemize}
        \item Kabel \emph{sollten} parallel zu Wänden verlaufen
        \item Hohlboden: Kabel haben wahrscheinlich genug Platz, um sich von Bohrspitze weg zu bewegen
        \item Betonboden: Sollten mindestens 5 cm dick sein, bevor Kabelrohr durchgeht.
    \end{itemize}
\end{itemize}




\section{Bauen}

\subsection{Bohren}

\paragraph{Decke}
\begin{itemize}
    \item Hohle Decke
    \item Putzdecke
    \item (Stahl-)Betondecke: braucht Schlagbohrer. Die meisten Zimmerdecken sind aus Stahlbeton.
\end{itemize}