\section{Data-structures}


\begin{table}[ht]
	\centering
	\caption{Most common data-structures}
	\begin{tabularx}{1.2\textwidth}{@{}rllll@{}}
		\toprule
		\multicolumn{1}{l}{}                      & \textbf{insertion}        & \textbf{removal}          & \textbf{search}           & \textbf{notes}            \\
		\cmidrule(l){2-5}
		\multicolumn{1}{r|}{\textbf{array}}       & x                         & x                         & n                         &                           \\
		\multicolumn{1}{r|}{\textbf{linked list}} & 1                         & 1                         & n                         & infinite size             \\
		\multicolumn{1}{r|}{\textbf{stack}}       & 1, always FILO            & 1, always FILO            & n                         &                           \\
		\multicolumn{1}{r|}{\textbf{queue}}       & 1, always FIFO            & 1, always FIFO            & n                         &                           \\
		\multicolumn{1}{r|}{\textbf{hash-table}}  & 1                         & 1                         & \begin{tabular}[c]{@{}l@{}}1 \\ n/k if table gets large\end{tabular} &                           \\
		\multicolumn{1}{r|}{\textbf{index tree}}  & log(n)                    & log(n)                    & log(n)                    &                           \\
		\multicolumn{1}{r|}{\textbf{binary heap}} & \begin{tabular}[c]{@{}l@{}}Average 1,\\ worst case log(n)\end{tabular} & \begin{tabular}[c]{@{}l@{}}Root element: 1\\ log(n) otherwise\end{tabular} & \begin{tabular}[c]{@{}l@{}}find-min: 1\\ Otherwise: n\end{tabular} & \begin{tabular}[c]{@{}l@{}}A binary tree where each child is larger than its parent\\ Stays semi-sorted: easy to get smallest element,\\ but increasingly harder to get second, third, ...\\ Useful for priority queues\end{tabular} \\
		\multicolumn{1}{r|}{\textbf{tries}}       &                           &                           &                           &                           \\
		\bottomrule
	\end{tabularx}
\end{table}


\subsection{Stack}

Stacks perform push and pop in O(1).
However, they do search in O(n).

\begin{lstlisting}[language=c]
#include <stdio.h>
#include <stdlib.h>
#include <string.h>!



typedef struct El {
	char* key;
	float val;
} El;


El* el_create(char* key, float val){
	char* keyk = strdup(key);
	El* el = malloc(sizeof(El));
	el->key = keyk;
	el->val = val;
	return el;
}


void el_destroy(El* el){
	free(el->key);
	free(el);
}


typedef struct Stack {
	El** data;
	int size;
	int top;
} Stack;


Stack* stack_create(int size){
	Stack* stack = malloc(sizeof(Stack));
	stack->data = malloc(sizeof(El*) * size);
	stack->size = size;
	stack->top = -1;
	return stack;
}

void stack_push_el(El* el, Stack* stack){
	stack->top += 1;
	stack->data[stack->top] = el;
}

void stack_push(char* key, float val, Stack* stack){
	El* el = el_create(key, val);
	stack_push_el(el, stack);
}

El* stack_pop(Stack* stack){
	stack->top -= 1;
	return stack->data[stack->top + 1];
}

El* stack_peak(Stack* stack){
	return stack->data[stack->top];
}

void stack_print(Stack* stack){
	int i;
	for(i = 0; i <= stack->top; i++){
		printf("%s -> %d", stack->data[i]->key, stack->data[i]->val);
	}
}

void stack_destroy(Stack* stack){
	while(stack->top >= 0){
		El* el = stack_pop(stack);
		el_destroy(el);
	}
	free(stack->data);
	free(stack);
}


int main(){

	Stack* s = stack_create(10);
	stack_push("eins", 1.1, s);
	stack_push("zwei", 1.2, s);
	stack_push("drei", 3.1, s);

	stack_print(s);

	stack_destroy(s);

	return 0;

}

\end{lstlisting}

\subsection{Linked lists}

Linked lists have the advantage that they have no prefixed size. Addition and insertion at any point is O(1). On the other hand, searching takes O(n) in the worst case.
\begin{lstlisting}[language=c]
#include <stdio.h>
#include <stdlib.h>
#include <string.h>

typedef struct String {
	char* text;
	int length;
} String;


String* string_create(char* text, int length){
	char* textCopy = strdup(text);
	String* string = malloc(sizeof(String));
	string->text = textCopy;
	string->length = length;
	return string;
}


void string_destroy(String* string){
	free(string->text);
	free(string);
}

typedef struct Node {
	struct String* data;
	struct Node* next;
} Node;


Node* node_create(String* text){
	Node* node = malloc(sizeof(Node));
	node->data = text;
	return node;
}

void node_destroy(Node* node){
	if (node->next != NULL) { node_destroy(node->next); }
	el_destroy(node->data);
	free(node);
}

void node_append(Node* ll, String* text) {
    if (ll->next != NULL) { ll_append(ll->next, text); }
    else {
        Node* newNode = node_create(text);
        ll->next = newNode;
    }
}

void node_print_all(Node* ll) {
    printf("%s \n", ll->data);
    if (ll->next != NULL) {
        node_print_all(ll->next);
    }
}

void main() {
    Node* ll = node_create(string_create("Hi!", 3));
    node_append(ll, string_create("I'm", 3));
    node_append(ll, string_create("Michael", 7));
    node_print_all(ll);
}
\end{lstlisting}


\subsection{Binary trees}

Trees make sense when the keys have to be ordered in some sense.

\begin{lstlisting}[language=c]
#include <stdio.h>
#include <stdlib.h>
#include <string.h>



typedef struct El {
	int key;
	float val;
} El;


El* el_create(int key, float val){
	El* el = malloc(sizeof(El));
	el->key = key;
	el->val = val;
	return el;
}


void el_destroy(El* el){
	free(el);
}

typedef struct Node {
	struct El* data;
	struct Node* parent;
	struct Node* left;
	struct Node* right;
} Node;


Node* node_create(Node* parent, El* element){
	Node* leaf = malloc(sizeof(Node));
	leaf->parent = parent;
	leaf->data = element;
	return leaf;
}

void node_destroy(Node* node){
	if(node->left != NULL){ node_destroy(node->left); }
	if(node->right != NULL){ node_destroy(node->right); }
	el_destroy(node->data);
	free(node);
}

void node_elementInsert(Node* node, El* el){
	if(el->key < node->data->key){
		if(node->left){
			node_elementInsert(node->left, el);
		}else{
			node->left = node_create(node, el);
		}
	}else{
		if(node->right){
			node_elementInsert(node->right, el);
		}else{
			node->right = node_create(node, el);
		}
	}
}

void node_dfsPrint(Node* node){
	printf("%d -> %d \n", node->data->key, node->data->val);
	if(node->left){
		node_dfsPrint(node->left);
	}
	if(node->right){
		node_dfsPrint(node->right);
	}

}

typedef struct Tree {
	Node* root;
} Tree;


Tree* tree_create(El* el){
	Tree* tree = malloc(sizeof(Tree));
	Node* root = node_create(NULL, el);
	tree->root = root;
	return tree;
}

void tree_destroy(Tree* tree){
	node_destroy(tree->root);
	free(tree);
}

void tree_elementInsert(Tree* tree, El* el){
	node_elementInsert(tree->root, el);
}

void tree_dfsPrint(Tree* tree){
	node_dfsPrint(tree->root);
}


int main(){

	El* baseEl = el_create(20, 14);
	El* el1 = el_create(1, 4);
	El* el2 = el_create(25, 9);
	El* el3 = el_create(2, 2);

	Tree* tree = tree_create(baseEl);

	tree_elementInsert(tree, el1);
	tree_elementInsert(tree, el2);
	tree_elementInsert(tree, el3);

	tree_dfsPrint(tree);

	tree_destroy(tree);
	return 0;

}

\end{lstlisting}

\subsection{Hash-tables}

Hash tables are basically a array with a hash-function. The hash function is there to associate string-keys to the integer-array-indices. The array needs as many elements as the hash-function can create integer-keys.

Through the hash-function the hash-table has insertion- and lookup-times of expected O(1). However, when the array gets many elements. it might happen that the hash-function gives the same integer-key to several string-keys. To acommodate such a case, the array-elements are implemented as linked lists. This makes the worst-case loopup-time O(n/k).


Why does one not simply use a perfect hash? Make a hash-function that generates a unique key for any string it's given. This would be known as a lookup-table. That works perfectly fine when we expect almost all possible string-keys to be really used. If however we  expect only a few of the possible strings to be used, the array would contain a lot of unused space.