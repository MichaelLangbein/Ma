\section{Backpropagation for hydrological parameter estimation}

Hydrological models consist of a network - a DAG - of landsurfaces, connected with flow paths. Each landsurface has one parameter $c$ (the flow-constant). 
It seems like to find optimal values for the $c$'s, we could exploit the network-structure of a catchment and apply the backpropagation algorithm.

\begin{itemize}
	\item Just like neural nets, we want to minimize an error function.
	\item Just like neural nets, we can do this by backpropagating the differential of a parameter. 
	\item Unlike neural nets, we don't consider all possible connections between nodes - the elevation-map can take that burden from us.
	\item Unlike neural nets, each node has state. This might complicate things - but in reality, when the net runs long enough, different initial states converge.
\end{itemize}


\subsection{Hydrological model}


\begin{lstlisting}[language=python]
class Node:
    def __init__(self, id, V0, c, a):
        self.id = id
        self.V = V0
        self.c = c
        self.a = a

    def eval(N, qIns):
        inpt = N*self.a + sum(qIns)
        outpt = self.c * self.V
        self.V += (inpt - outpt)
        return outpt


class Model:
    def __init__(self, nodes, connections):
        self.nodes = nodes              #{id: node}
        self.connections = connections  #[(idFrom, idTo)]

    def eval(Ns):    #Ns: {id: n}
        lastNode = self.nodes[0]
        qOut = evalNode(lastNode, Ns)
        return qOut

    def evalNode(self, node, Ns):
        upstreamNeighbours = self.getUpstreamNeighbours(node)
        qs = []
        for neighbour in upstreamNeighbours:
            qUpstream = self.evalNode(neighbour, Ns)
            qs.append(qUpstream)
        qOut = node.eval(Ns[node.id], qs)
        return qOut

    def getUpstreamNeighbours(self, node):
        neighbours = []
        for connection in self.connections:
            if connection[1] == node.id:
                neighbourId = connection[0]
                neighbour = self.nodes[neighbourId]
                neighbours.append(neighbour)
        return neighbours

    def gradients(self, sse):
        pass

    def updateParas(self, gradients):
        pass

    
            
def modelFactory(terrainMap):
    nodes = None
    connections = None
    model = Model(nodes, connections)
    return model

def optimize(model, N_timeSeries, q_timeSeries):
    errors = []
    for t, Ns in enumerate(N_timeSeries):
        qPred = model.eval(Ns)
        qObs = q_timeSeries[t]
        errors.append([qPred - qObs]**2)
    sse = sum(errors)
    grads = model.gradients(sse)
    model.update(grads)




terrainMap = None
N_timeSeries = None
q_timeSeries = None
model = modelFactory(terrainMap)
optimize(model, N_timeSeries, q_timeSeries)
\end{lstlisting}


\subsection{Optimisation using tensorflow}



\section{Hydrological models as Hidden Markov Models}
Whereas the above section dealt primarily with parameter estimation, here we are going to handle state estimation. 
