\section{Backpropagation for hydrological parameter estimation}

Hydrological models consist of a network - a DAG - of landsurfaces, connected with flow paths. Each landsurface has one parameter $c$ (the flow-constant). 
It seems like to find optimal values for the $c$'s, we could exploit the network-structure of a catchment and apply the backpropagation algorithm.

\begin{itemize}
	\item Just like neuronal nets, we want to minimize an error function.
	\item Just like neuronal nets, we can do this by backpropagating the differential of a parameter. 
	\item Unlike neural nets, we don't consider all possible connections between nodes - the elevation-map can take that burden from us.
	\item Unlike neural nets, each node has state. This might complicate things - but in reality, when the net runs long enough, different initial states converge.
\end{itemize}


\subsection{Hydrological model}

\subsection{Optimisation using tensorflow}



\section{Hydrological models as Hidden Markov Models}
Whereas the above section dealt primarily with parameter estimation, here we are going to handle state estimation. 