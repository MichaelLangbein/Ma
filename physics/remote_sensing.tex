\section{Remote sensing}

\subsection{Electromagnetic radiation}



\subsection{Geospatial processing}

\subsubsection{Datatypes and protocols}

\paragraph{Datacubes} are tensors of (usually very large) (geo-)data.

\paragraph{OGC-WCPS: OGC's Web Coverage Processing Service} is a language for filtering and processing multi-dimensional raster coverages. 

\paragraph{Open-EO} ia an API to speak to different geo-processing services like GEE. Unfortunately, open-EO stands in concurrence to WCPS. 



\subsection{Important satellites}

\subsubsection{Landsat}
The Landsat program is the longest-running enterprise for acquisition of satellite imagery of Earth. On July 23, 1972 the Earth Resources Technology Satellite was launched. This was eventually renamed to Landsat.[1] The most recent, Landsat 8, was launched on February 11, 2013. The instruments on the Landsat satellites have acquired millions of images. The images, archived in the United States and at Landsat receiving stations around the world, are a unique resource for global change research and applications in agriculture, cartography, geology, forestry, regional planning, surveillance and education, and can be viewed through the U.S. Geological Survey (USGS) 'EarthExplorer' website. Landsat 7 data has eight spectral bands with spatial resolutions ranging from 15 to 60 meters; the temporal resolution is 16 days.[2] Landsat images are usually divided into scenes for easy downloading. Each Landsat scene is about 115 miles long and 115 miles wide (or 100 nautical miles long and 100 nautical miles wide, or 185 kilometers long and 185 kilometers wide).

\subsubsection{Modis}

\subsubsection{Copernicus}

\subsubsection{Sentinel}
ESA is currently developing seven missions under the Sentinel programme. The Sentinel missions include radar and super-spectral imaging for land, ocean and atmospheric monitoring. Each Sentinel mission is based on a constellation of two satellites to fulfill and revisit the coverage requirements for each mission, providing robust datasets for all Copernicus services.


\subsection{Important service providers}

\paragraph{CHIRPS} is ...

\paragraph{EODC}: \href{https://www.eodc.eu/}{Earth Observation Data Center for Water Resources Monitoring}

\paragraph{Eurac}: \href{http://www.eurac.edu}{Eurac} is a private research company ...

\paragraph{Google Earth Engine} provides ...

\paragraph{NASA's ECS} (Earth observation center Core System) is a vast catalogue of ...