\section{Graphics Programming}


\subsection{Rendering in the browser}
\begin{table}[]
    \begin{tabular}{lllll}
        \hline
        technology   & use-case                    & programming style &                                                                                                                                                                 & libraries           \\ \hline
        canvas/2d    &                             &                   & single-buffering to bitmap. paintbrush-statemachine.                                                                                                            &                     \\
        canvas/webgl & 3d, interactive, many items & procedural        & double-buffering to bitmap                                                                                                                                      & threejs, processing \\
        svg + css    & simple 2d graphics          & declarative       & no bitmap, but \textbackslash{}inlinecode\{\textless{}svg\textgreater{}\}. has concept of layers, browser-native event-handling. Cannot easily export to image. & d3, raphael         \\ \hline
    \end{tabular}
\end{table}


\subsection{SVG}

SVG consists of 
\begin{itemize}
    \item Objects
    \item Groups. Groups have ...
        \begin{itemize}
            \item a transform
            \item a style
            \item Notably, they don't have a x, y, width or height attribute! If you need those, use SVG's instead
        \end{itemize}
    \item SVG's. 
        \begin{itemize}
            \item x, y width, height
        \end{itemize}
\end{itemize}

Objects have a nice, simple hierarchy.
\begin{itemize}
    \item Objects: each object is a ...
        \begin{itemize}
            \item Path: the most basic object: a Bezier-curve. Each of the following subtypes can be downgraded to a path again in inkscape.
                \begin{itemize}
                    \item rect, star, ellipse, text
                \end{itemize}
        \end{itemize}
    \item Each object has a ... 
        \begin{itemize}
            \item fill, stroke, opacity
        \end{itemize}
\end{itemize}

A bunch of software helps with SVG's:
\begin{itemize}
    \item Inkscape: ideal for creating SVG logos
    \item d3: Data-driven, animated SVG's
    \item Raphael: animated SVG's; imperative, \emph{not} data-driven
\end{itemize}


\subsection{D3}

\subsection{WASM}

\begin{itemize}
    \item Module: compiled WASM. Can be passed around. Stateless.
    \item Instance: a module + it's state
    \item State: \begin{itemize}
        \item Memory: resizable ArrayBuffer
        \item Table: array of references to functions
    \end{itemize}
\end{itemize}

\begin{itemize}
    \item Calls from js to wasm are synchronous.
\end{itemize}

\begin{itemize}
    \item LLVM: compiles C to intermediate (via the Clang-Frontend) and intermediate to WASM (via the WASM-Backend)
    \item Emscripten: includes LLVM. But also creates custom JS to be used by WASM to emulate \inlinecode{stdio} et.al.
\end{itemize}


\subsection{Webgl}

\begin{lstlisting}
    import { flattenMatrix } from './engine.shapes';

    /**
     * WEBGL
     *
     * A rasterization engine that allows to draw points, line segments, or triangles.
     *
     * Vertex shaders take whatever coordinates you use and return a 3-d array with elements between -1 and 1.
     * Basically, this is a 3d-array, but WebGl does not use the z-axis for real perspective, but only to differentiate
     * what pixel lies in front of another.
     * This is not like looking in a 3d-box, but rather like looking on multiple stacked sheets on a projector.
     * Actually, this is a lie. WebGl uses 4 coordinates: x, y, z and w. The above only holds if you keep w at 1.
     * After applying the vertex shader, WebGl divides all coordinates by w, yielding (x/w, y/w, z/w, 1).
     * This can be used to calculate projections - google for 'homogeneous coordinates' to find out more.
     * Compare this [site](https://www.tomdalling.com/blog/modern-opengl/explaining-homogenous-coordinates-and-projective-geometry/)
     * and the shader `basic3d.vert.glsl`.
     *
     * WebGL knows two data structures:
     *  - buffers (generic byte arrays): usually positions, normals, texture-coordinates, vertex-colors etc.
     *    buffers are accessed in shaders as 'attributes'.
     *    note that buffers contain one entry for each vertex.
     *  - textures (bitmap images).
     *
     * Shaders use these data structures in two different ways.
     *  - Attributes are values, one per vertex.
     *    For the shader, attributes are read-only.
     *    Attributes default to [0, 0, 0, 1]
     *  - Uniforms are values, one per shader.
     *    For the shader, uniforms are read-only.
     *
     * Apart from this, shaders know about two more types of data:
     *  - Varyings are values that are passed from vertex-shader to fragment-shader.
     *    They are read-only only for the fragment-shader.
     *  - Const: a compile-time constant.
     *
     * A program is just a list of compiled and linked vertex- and fragment-shaders.
     *
     *
     * Drawing: there's drawArrays and drawElements.
     *  - drawArrays is the robust all-rounder.
     *  - drawElements can be more performant if you share vertices between objects.
     *
     *
     * Rendering data is fast, but uploading it into GPU memory is slow.
     * Optimizing WebGl performance mostly means: Avoiding having GPU and CPU wait for each other.
     * The more the GPU can do in bulk, the better. The more often you have to upload data from CPU to GPU, the worse.
     *  - So avoid switching programs, buffers and uniforms if you can.
     *    (You won't be able to avoid switching buffers, because every object is likely different. But sort your objects by their shaders, and you'll save a lot of time.)
     *  - Try to do translations, rotations and shears inside the vertex-shader instead of altering the object's buffer.
     *  - If appropriate, create über-shaders and über-buffers, that contain information for more than just one object.
     *
     * There is another thing that affects performance:
     * WebGL will only run fragment-shaders when the object's pixels aren't already obscured by a larger object in front of it.
     * That means it makes sense to first draw large objects that are close to the camera - all objects behind them won't need their fragment-shader executed.
     *
     * All `create*` functions unbind variables after setting their values. This is to avoid unwanted side-effects.
     */
    
    
    
    const shaderInputTextureBindPoint = 0;
    const textureConstructionBindPoint = 7;
    
    
    
    
    /**
     * Compile shader.
     */
    export const compileShader = (gl: WebGLRenderingContext, typeBit: number, shaderSource: string): WebGLShader => {
        const shader = gl.createShader(typeBit);
        if (!shader) {
            throw new Error('No shader was created');
        }
        gl.shaderSource(shader, shaderSource);
        gl.compileShader(shader);
        if (!gl.getShaderParameter(shader, gl.COMPILE_STATUS)) {
            gl.deleteShader(shader);
            throw new Error(`An error occurred compiling the shader: ${gl.getShaderInfoLog(shader)}.    \n\n Shader code: ${shaderSource}`);
        }
        return shader;
    };
    
    
    /**
     * Note that every program *must* have one and only one vertex-shader
     * and one and only one fragment shader.
     * That means you cannot add multiple fragment-shaders in one program. Instead, either load them in consecutively as part of different programs,
     * or generate an über-shader that contains both codes.
     */
    export const createShaderProgram = (gl: WebGLRenderingContext, vertexShaderSource: string, fragmentShaderSource: string): WebGLProgram => {
    
        const program = gl.createProgram();
        if (!program) {
            throw new Error('No program was created');
        }
    
        const vertexShader = compileShader(gl, gl.VERTEX_SHADER, vertexShaderSource);
        const fragmentShader = compileShader(gl, gl.FRAGMENT_SHADER, fragmentShaderSource);
        gl.attachShader(program, vertexShader);
        gl.attachShader(program, fragmentShader);
    
        gl.linkProgram(program);
    
        gl.detachShader(program, vertexShader);
        gl.detachShader(program, fragmentShader);
        gl.deleteShader(vertexShader);
        gl.deleteShader(fragmentShader);
    
        if (!gl.getProgramParameter(program, gl.LINK_STATUS)) {
            gl.deleteProgram(program);
            throw new Error('Unable to initialize the shader program: ' + gl.getProgramInfoLog(program));
        }
    
        return program;
    };
    
    
    export const setup3dScene = (gl: WebGLRenderingContext): void => {
        gl.viewport(0, 0, gl.canvas.width, gl.canvas.height);  // making sure that shader-coordinate-system goes from 0 to 1.
    
        gl.enable(gl.DEPTH_TEST);
        gl.depthFunc(gl.LEQUAL);
        gl.cullFace(gl.BACK);
    
        clearBackground(gl, [0, 0, 0, 1]);
    };
    
    
    export const bindProgram = (gl: WebGLRenderingContext, program: WebGLProgram): void => {
        gl.useProgram(program);
    };
    
    
    export const clearBackground = (gl: WebGLRenderingContext, color: number[]): void => {
        gl.clearColor(color[0], color[1], color[2], color[3]);
        gl.clearDepth(1.0);
        gl.clear(gl.COLOR_BUFFER_BIT | gl.DEPTH_BUFFER_BIT);
    };
    
    
     /**
      * A generic buffer, together with it's metadata.
      */
    export interface BufferObject {
        buffer: WebGLBuffer;
        vectorSize: number;
        vectorCount: number;
        type: number;
        normalize: boolean;
        stride: number;
        offset: number;
        drawingMode: number; // gl.TRIANGLES, gl.POINTS, or gl.LINES
    }
    
    
    /**
     * Create buffer. Creation is slow! Do *before* render loop.
     */
    export const createFloatBuffer = (gl: WebGLRenderingContext, data: number[][], drawingMode: number = gl.TRIANGLES): BufferObject => {
    
        const dataFlattened = new Float32Array(flattenMatrix(data));
    
        const buffer = gl.createBuffer();
        if (!buffer) {
            throw new Error('No buffer was created');
        }
        gl.bindBuffer(gl.ARRAY_BUFFER, buffer);
        gl.bufferData(gl.ARRAY_BUFFER, dataFlattened, gl.STATIC_DRAW);
        // STATIC_DRAW: tells WebGl that we are not likely to change this data much.
        gl.bindBuffer(gl.ARRAY_BUFFER, null);  // unbinding
    
        const bufferObject: BufferObject = {
            buffer: buffer,
            vectorSize: data[0].length,
            vectorCount: data.length,
            type: gl.FLOAT,   // the data is 32bit floats
            normalize: false, // don't normalize the data
            stride: 0,        // 0 = move forward size * sizeof(type) each iteration to get the next position. Only change this in very-high-performance jobs.
            offset: 0,        // start at the beginning of the buffer. Only change this in very-high-performance jobs.
            drawingMode: drawingMode
        };
    
        return bufferObject;
    };
    
    
    export const drawArray = (gl: WebGLRenderingContext, bo: BufferObject): void => {
        gl.drawArrays(bo.drawingMode, bo.offset, bo.vectorCount);
    };
    
    
    
    export const updateBufferData = (gl: WebGLRenderingContext, bo: BufferObject, newData: number[][]): BufferObject => {
    
        const dataFlattened = new Float32Array(flattenMatrix(newData));
    
        gl.bindBuffer(gl.ARRAY_BUFFER, bo.buffer);
        gl.bufferData(gl.ARRAY_BUFFER, dataFlattened, gl.STATIC_DRAW);
        gl.bindBuffer(gl.ARRAY_BUFFER, null);  // unbinding
    
        const newBufferObject: BufferObject = {
            buffer: bo.buffer,
            vectorSize: newData[0].length,
            vectorCount: newData.length,
            type: gl.FLOAT,   // the data is 32bit floats
            normalize: false, // don't normalize the data
            stride: 0,        // 0 = move forward size * sizeof(type) each iteration to get the next position. Only change this in very-high-performance jobs.
            offset: 0,        // start at the beginning of the buffer. Only change this in very-high-performance jobs.
            drawingMode: bo.drawingMode,
        };
    
        return newBufferObject;
    };
    
    
    
    
    /**
     * Fetch attribute's location (attribute declared in some shader). Slow! Do *before* render loop.
     */
    export const getAttributeLocation = (gl: WebGLRenderingContext, program: WebGLProgram, attributeName: string): number => {
        const loc = gl.getAttribLocation(program, attributeName);
        if (loc === -1) {
            throw new Error(`Couldn't find attribute ${attributeName} in program.`);
        }
        return loc;
    };
    
    
    
    /**
     * Attributes vary from vertex to vertex - that means that there are *many* of them.
     * So it makes sense for WebGl to store attribute values in a dedicated data structure - the buffer.
     */
    export const bindBufferToAttribute = (gl: WebGLRenderingContext, attributeLocation: number, bufferObject: BufferObject): void => {
        // Enable editing
        gl.enableVertexAttribArray(attributeLocation);
        // Bind buffer to ARRAY_BUFFER
        gl.bindBuffer(gl.ARRAY_BUFFER, bufferObject.buffer);
        // Bind the buffer currently at ARRAY_BUFFER to a vertex-buffer-location.
        gl.vertexAttribPointer(
            attributeLocation,
            bufferObject.vectorSize, bufferObject.type, bufferObject.normalize, bufferObject.stride, bufferObject.offset);
        // gl.disableVertexAttribArray(attributeLocation); <-- must not do this!
    };
    
    
    
    
    export interface IndexBufferObject {
        buffer: WebGLBuffer;
        count: number;
        type: number; // must be gl.UNSIGNED_SHORT
        offset: number;
        drawingMode: number; // gl.TRIANGLES, gl.POINTS, or gl.LINES
    }
    
    export const createIndexBuffer = (gl: WebGLRenderingContext, indices: number[][], drawingMode: number = gl.TRIANGLES): IndexBufferObject => {
    
        const indicesFlattened = new Uint16Array(flattenMatrix(indices));
    
        const buffer = gl.createBuffer();
        if (!buffer) {
            throw new Error('No buffer was created');
        }
        gl.bindBuffer(gl.ELEMENT_ARRAY_BUFFER, buffer);
        gl.bufferData(gl.ELEMENT_ARRAY_BUFFER, indicesFlattened, gl.STATIC_DRAW);
        gl.bindBuffer(gl.ELEMENT_ARRAY_BUFFER, null);
    
        const bufferObject: IndexBufferObject = {
            buffer: buffer,
            count: indicesFlattened.length,
            type: gl.UNSIGNED_SHORT,
            offset: 0,
            drawingMode: drawingMode
        };
    
        return bufferObject;
    };
    
    export const drawElements = (gl: WebGLRenderingContext, ibo: IndexBufferObject): void => {
        gl.bindBuffer(gl.ELEMENT_ARRAY_BUFFER, ibo.buffer);  // @TODO: optimize this to somewhere else?
        gl.drawElements(ibo.drawingMode, ibo.count, ibo.type, ibo.offset);
    };
    
    
    export const draw = (gl: WebGLRenderingContext, b: BufferObject | IndexBufferObject): void => {
        if (b.type  === gl.UNSIGNED_SHORT) {
            drawElements(gl, b as IndexBufferObject);
        } else {
            drawArray(gl, b as BufferObject);
        }
    };
    
    
    
    
    
    export interface TextureObject {
        texture: WebGLTexture;
        width: number;
        height: number;
        level: number;
        internalformat: number;
        format: number;
        type: number;
    }
    
    /**
     * A shader's attributes get their buffer-values from the VERTEX_ARRAY, but they are constructed in the ARRAY_BUFFER.
     * Textures analogously are served from the TEXTURE_UNITS, while for construction they are bound to ACTIVE_TEXTURE.
     *
     * There is a big difference, however. Contrary to buffers which receive their initial value while still outside the ARRAY_BUFFER,
     * a texture does already have to be bound into the TEXTURE_UNITS when it's being created.
     * Since it'll always be bound into the slot that ACTIVE_TEXTURE points to, you can inadvertently overwrite another texture that is
     * currently in this place. To avoid this, we provide a dedicated `textureConstructionBindPoint`.
     *
     * Buffers are easier in this, since with vertexAttribPointer we are guaranteed to get a slot in the VERTEX_ARRAY that is not
     * already occupied by another buffer.
     */
    export const createTexture = (gl: WebGLRenderingContext, image: HTMLImageElement | HTMLCanvasElement): TextureObject => {
    
        const texture = gl.createTexture();  // analog to createBuffer
        if (!texture) {
            throw new Error('No texture was created');
        }
        gl.activeTexture(gl.TEXTURE0 + textureConstructionBindPoint); // so that we don't overwrite another texture in the next line.
        gl.bindTexture(gl.TEXTURE_2D, texture);  // analog to bindBuffer. Binds texture to currently active texture-bindpoint (aka. texture unit).
        gl.texImage2D(gl.TEXTURE_2D, 0, gl.RGBA, gl.RGBA, gl.UNSIGNED_BYTE, image);  // analog to bufferData
        gl.generateMipmap(gl.TEXTURE_2D); // mipmaps are mini-versions of the texture.
        gl.bindTexture(gl.TEXTURE_2D, null);  // unbinding
    
        let w, h: number;
        if (image instanceof HTMLImageElement) {
            w = image.naturalWidth;
            h = image.naturalHeight;
        } else {
            w = image.width;
            h = image.height;
        }
    
        const textureObj: TextureObject = {
            texture: texture,
            level: 0,
            internalformat: gl.RGBA,
            format: gl.RGBA,
            type: gl.UNSIGNED_BYTE,
            width: w,
            height: h
        };
    
        return textureObj;
    };
    
    
    
    export const createEmptyTexture = (gl: WebGLRenderingContext, width: number, height: number): TextureObject => {
        if (width <= 0 || height <= 0) {
            throw new Error('Width and height must be positive.');
        }
        const texture = gl.createTexture();
        if (!texture) {
            throw new Error('No texture was created');
        }
        gl.activeTexture(gl.TEXTURE0 + textureConstructionBindPoint); // so that we don't overwrite another texture in the next line.
        gl.bindTexture(gl.TEXTURE_2D, texture);
        gl.texImage2D(gl.TEXTURE_2D, 0, gl.RGBA, width, height, 0, gl.RGBA, gl.UNSIGNED_BYTE, null);
        gl.texParameteri(gl.TEXTURE_2D, gl.TEXTURE_WRAP_S, gl.CLAMP_TO_EDGE);
        gl.texParameteri(gl.TEXTURE_2D, gl.TEXTURE_WRAP_T, gl.CLAMP_TO_EDGE);
        gl.texParameteri(gl.TEXTURE_2D, gl.TEXTURE_MIN_FILTER, gl.NEAREST);
        gl.texParameteri(gl.TEXTURE_2D, gl.TEXTURE_MAG_FILTER, gl.NEAREST);
        gl.bindTexture(gl.TEXTURE_2D, null);
    
        const textureObj: TextureObject = {
            texture: texture,
            level: 0,
            internalformat: gl.RGBA,
            format: gl.RGBA,
            type: gl.UNSIGNED_BYTE,
            width: width,
            height: height
        };
    
        return textureObj;
    };
    
    
    /**
     * Even though we reference textures as uniforms in a fragment shader, assigning an actual texture-value to that uniform works differently than for normal uniforms.
     * Normal uniforms have a concrete value.
     * Texture uniforms, on the other hand, are just an integer-index that points to a special slot in the GPU memory (the bindPoint) where the actual texture value lies.
     */
    export const bindTextureToUniform = (gl: WebGLRenderingContext, texture: WebGLTexture, bindPoint: number, uniformLocation: WebGLUniformLocation): void =>  {
        if (bindPoint > gl.getParameter(gl.MAX_COMBINED_TEXTURE_IMAGE_UNITS)) {
            throw new Error(`There are only ${gl.getParameter(gl.MAX_COMBINED_TEXTURE_IMAGE_UNITS)} texture bind points, but you tried to bind to point nr. ${bindPoint}.`);
        }
        if (bindPoint === textureConstructionBindPoint) {
            console.error(`You are about to bind to the dedicated texture-construction bind point (nr. ${bindPoint}).
            If after this call another texture is built, your shader will now use that new texture instead of this one!
            Consider using another bind point.`);
        }
        gl.activeTexture(gl.TEXTURE0 + bindPoint);  // analog to enableVertexAttribArray
        gl.bindTexture(gl.TEXTURE_2D, texture);  // analog to bindBuffer. Binds texture to currently active texture-bindpoint (aka. texture unit).
        gl.uniform1i(uniformLocation, bindPoint); // analog to vertexAttribPointer
    };
    
    
    
    export const updateTexture = (gl: WebGLRenderingContext, to: TextureObject, newImage: HTMLImageElement | HTMLCanvasElement): TextureObject => {
    
        gl.activeTexture(gl.TEXTURE0 + textureConstructionBindPoint); // so that we don't overwrite another texture in the next line.
        gl.bindTexture(gl.TEXTURE_2D, to.texture);  // analog to bindBuffer. Binds texture to currently active texture-bindpoint (aka. texture unit).
        gl.texImage2D(gl.TEXTURE_2D, 0, gl.RGBA, gl.RGBA, gl.UNSIGNED_BYTE, newImage);  // analog to bufferData
        gl.generateMipmap(gl.TEXTURE_2D); // mipmaps are mini-versions of the texture.
        gl.bindTexture(gl.TEXTURE_2D, null);  // unbinding
    
        let w, h: number;
        if (newImage instanceof HTMLImageElement) {
            w = newImage.naturalWidth;
            h = newImage.naturalHeight;
        } else {
            w = newImage.width;
            h = newImage.height;
        }
        to.width = w;
        to.height = h;
    
        return to;
    };
    
    
    export interface FramebufferObject {
        framebuffer: WebGLFramebuffer;
        texture: TextureObject;
        width: number;
        height: number;
    }
    
    
    export const createFramebuffer = (gl: WebGLRenderingContext): WebGLFramebuffer => {
        const fb = gl.createFramebuffer();  // analog to createBuffer
        if (!fb) {
            throw new Error(`Error creating framebuffer`);
        }
        return fb;
    };
    
    
    /**
     * The operations `clear`, `drawArrays` and `drawElements` only affect the currently bound framebuffer.
     */
    export const bindFramebuffer = (gl: WebGLRenderingContext, fbo: FramebufferObject) => {
        gl.bindFramebuffer(gl.FRAMEBUFFER, fbo.framebuffer);
        gl.viewport(0, 0, fbo.width, fbo.height);
        // Note that binding the framebuffer does *not* mean binding its texture. In fact, if there is a bound texture, it must be the *input* to a shader, not the output.
        // Therefore, a framebuffer's texture must not be bound when the framebuffer is.
    };
    
    
    export const bindOutputCanvasToFramebuffer = (gl: WebGLRenderingContext) => {
        gl.bindFramebuffer(gl.FRAMEBUFFER, null);
        gl.viewport(0, 0, gl.canvas.width, gl.canvas.height);
    };
    
    
    /**
     * A framebuffer can have a texture - that is the bitmap that the shader-*out*put is drawn on.
     * Shaders may also have one or more *in*put texture(s), which must be provided to the shader as a uniform sampler2D.
     * Only the shader needs to know about any potential input texture, the framebuffer will always only know about it's output texture.
     */
    export const bindTextureToFramebuffer = (gl: WebGLRenderingContext, texture: TextureObject, fb: WebGLFramebuffer): FramebufferObject => {
        gl.bindFramebuffer(gl.FRAMEBUFFER, fb);
        gl.framebufferTexture2D(gl.FRAMEBUFFER, gl.COLOR_ATTACHMENT0, gl.TEXTURE_2D, texture.texture, 0); // analog to bufferData
    
        if (gl.checkFramebufferStatus(gl.FRAMEBUFFER) !== gl.FRAMEBUFFER_COMPLETE) {
            throw new Error(`Error creating framebuffer: framebuffer-status: ${gl.checkFramebufferStatus(gl.FRAMEBUFFER)} ; error-code: ${gl.getError()}`);
        }
    
        gl.bindFramebuffer(gl.FRAMEBUFFER, null);
    
        const fbo: FramebufferObject = {
            framebuffer: fb,
            texture: texture,
            width: texture.width,
            height: texture.height
        };
    
        return fbo;
    };
    
    
    
    
    
    
    
    
    
    
    /**
     * Fetch uniform's location (uniform declared in some shader). Slow! Do *before* render loop.
     */
    export const getUniformLocation = (gl: WebGLRenderingContext, program: WebGLProgram, uniformName: string): WebGLUniformLocation => {
        const loc = gl.getUniformLocation(program, uniformName);
        if (loc === null) {
            throw new Error(`Couldn't find uniform ${uniformName} in program.`);
        }
        return loc;
    };
    
    
    
    
    export type UniformType = '1i' | '2i' | '3i' | '4i' | '1f' | '2f' | '3f' | '4f' | '1fv' | '2fv' | '3fv' | '4fv' | 'matrix2fv' | 'matrix3fv' | 'matrix4fv';
    
    /**
     * Contrary to attributes, uniforms don't need to be stored in a buffer.
     *
     * 'v' is not about the shader, but how you provide data from the js-side.
     * uniform1fv(loc, [3.19]) === uniform1f(loc, 3.19)
     *
     * |js                                      |          shader                  |
     * |----------------------------------------|----------------------------------|
     * |uniform1f(loc, 3.19)                    |  uniform float u_pi;             |
     * |uniform2f(loc, 3.19, 2.72)              |  uniform vec2 u_constants;       |
     * |uniform2fv(loc, [3.19, 2.72])           |  uniform vec2 u_constants;       |
     * |uniform1fv(loc, [1, 2, 3, 4, 5, 6])     |  uniform float u_kernel[6];      |
     * |uniform2fv(loc, [1, 2, 3, 4, 5, 6])     |  uniform vec2 u_observations[3]; |
     * |uniformMatrix3fv(loc, [[...], [...]])   |  uniform mat3 u_matrix;          |
     *
     */
    export const bindValueToUniform = (gl: WebGLRenderingContext, uniformLocation: WebGLUniformLocation, type: UniformType, values: number[]): void => {
        switch (type) {
            case '1i':
                gl.uniform1i(uniformLocation, values[0]);
                break;
    
            case '1f':
                gl.uniform1f(uniformLocation, values[0]);
                break;
            case '2f':
                gl.uniform2f(uniformLocation, values[0], values[1]);
                break;
            case '3f':
                gl.uniform3f(uniformLocation, values[0], values[1], values[2]);
                break;
            case '4f':
                gl.uniform4f(uniformLocation, values[0], values[1], values[2], values[3]);
                break;
    
            case '1fv':
                gl.uniform1fv(uniformLocation, values);
                break;
            case '2fv':
                gl.uniform2fv(uniformLocation, values);
                break;
            case '3fv':
                gl.uniform3fv(uniformLocation, values);
                break;
            case '4fv':
                gl.uniform4fv(uniformLocation, values);
                break;
    
            // In the following *matrix* calls, the 'transpose' parameter must always be false. 
            // Quoting the OpenGL ES 2.0 spec:
            // If the transpose parameter to any of the UniformMatrix* commands is
            // not FALSE, an INVALID_VALUE error is generated, and no uniform values are
            // changed.
            case 'matrix2fv':
                gl.uniformMatrix2fv(uniformLocation, false, values);
                break;
    
            case 'matrix3fv':
                gl.uniformMatrix3fv(uniformLocation, false, values);
                break;
    
            case 'matrix4fv':
                gl.uniformMatrix4fv(uniformLocation, false, values);
                break;
    
            default:
                throw Error(`Type ${type} not yet implemented.`);
        }
    };
    
    
    /**
     * (From https://hacks.mozilla.org/2013/04/the-concepts-of-webgl/ and https://stackoverflow.com/questions/56303648/webgl-rendering-buffers:)
     * Ignoring handmade framebuffers, WebGl has two framebuffers that are always in use: the `frontbuffer/displaybuffer` and the `backbuffer/drawingbuffer`.
     * WebGl per default renders to the `drawingbuffer`, aka. the `backbuffer`.
     * There is also the currently displayed buffer, named the `frontbuffer` aka. the `displaybuffer`.
     * the WebGL programmer has no explicit access to the frontbuffer whatsoever.
     *
     * Once you called `clear`, `drawElements` or `drawArrays`, the browser marks the canvas as `needs to be composited`.
     * (Assuming `preserveDrawingBuffer == false`:) Immediately before compositing, the browser
     *  - swaps the back- and frontbuffer
     *  - clears the new backbuffer.
     * (If `preserveDrawingBuffer === true`: ) Immediately before compositing, the browser
     *  - copies the drawingbuffer to the frontbuffer.
     *
     * As a consequence, if you're going to use canvas.toDataURL or canvas.toBlob or gl.readPixels or any other way of getting data from a WebGL canvas,
     * unless you read it in the same event then it will likely be clear when you try to read it.
     *
     * In the past, old games always preserved the drawing buffer, so they'd only have to change those pixels that have actually changed. Nowadays preserveDrawingBuffer is false by default.
     *
     * A (almost brutal) workaround to get the canvas to preserve the drawingBuffer can be found here: https://stackoverflow.com/questions/26783586/canvas-todataurl-returns-blank-image
     */
    export const getCurrentFramebuffersPixels = (canvas: HTMLCanvasElement): ArrayBuffer  => {
        const gl = canvas.getContext('webgl');
        if (!gl) {
            throw new Error('no context');
        }
    
        const format = gl.getParameter(gl.IMPLEMENTATION_COLOR_READ_FORMAT);
        const type = gl.getParameter(gl.IMPLEMENTATION_COLOR_READ_TYPE);
    
        let pixels;
        if (type === gl.UNSIGNED_BYTE) {
            pixels = new Uint8Array(gl.drawingBufferWidth * gl.drawingBufferHeight * 4);
        } else if (type === gl.UNSIGNED_SHORT_5_6_5 || type === gl.UNSIGNED_SHORT_4_4_4_4 || type === gl.UNSIGNED_SHORT_5_5_5_1) {
            pixels = new Uint16Array(gl.drawingBufferWidth * gl.drawingBufferHeight * 4);
        } else if (type === gl.FLOAT) {
            pixels = new Float32Array(gl.drawingBufferWidth * gl.drawingBufferHeight * 4);
        } else {
            throw new Error(`Did not understand pixel data type ${type} for format ${format}`);
        }
    
        // Just like `toDataURL` or `toBlob`, `readPixels` does not access the frontbuffer.
        // It accesses the backbuffer or any other currently active framebuffer.
        gl.readPixels(0, 0, canvas.width, canvas.height, format, type, pixels);
    
        return pixels;
    };        
\end{lstlisting}

\begin{lstlisting}
    import { createShaderProgram, setup3dScene, createFloatBuffer, getAttributeLocation, bindBufferToAttribute, getUniformLocation, bindValueToUniform, clearBackground, BufferObject, UniformType, bindProgram, createTexture, bindTextureToUniform, TextureObject, FramebufferObject, bindFramebuffer, bindOutputCanvasToFramebuffer, updateBufferData, bindTextureToFramebuffer, createEmptyTexture, createFramebuffer, updateTexture, createIndexBuffer, IndexBufferObject, drawArray } from './webgl';


// dead-simple hash function - not intended to be secure in any way.
const hash = function(s: string): string {
    let h = 0;
    for (const c of s) {
        h += c.charCodeAt(0);
    }
    return `${h}`;
};


export interface IProgram {
    program: WebGLProgram;
    id: string;
    vertexShaderSource: string;
    fragmentShaderSource: string;
}


export class Program implements IProgram {

    readonly program: WebGLProgram;
    readonly id: string;

    constructor(gl: WebGLRenderingContext,
        readonly vertexShaderSource: string,
        readonly fragmentShaderSource: string) {
        this.program = createShaderProgram(gl, vertexShaderSource, fragmentShaderSource);
        this.id = hash(vertexShaderSource + fragmentShaderSource);
    }
}


export interface IUniform {
    location: WebGLUniformLocation;
    type: UniformType;
    value: number[];
    variableName: string;
}


export class Uniform implements IUniform {

    readonly location: WebGLUniformLocation;
    readonly type: UniformType;
    readonly value: number[];
    readonly variableName: string;

    constructor(gl: WebGLRenderingContext, program: IProgram, variableName: string, type: UniformType, data: number[]) {
        this.location = getUniformLocation(gl, program.program, variableName);
        this.type = type;
        this.value = data;
        this.variableName = variableName;
    }
}


export interface ITexture {
    location: WebGLUniformLocation;
    bindPoint: number;
    texture: TextureObject;
    variableName: string;
}

export class Texture implements ITexture {

    readonly location: WebGLUniformLocation;
    readonly bindPoint: number;
    readonly texture: TextureObject;
    readonly variableName: string;

    constructor(gl: WebGLRenderingContext, program: IProgram, variableName: string, im: HTMLImageElement | HTMLCanvasElement | TextureObject, bindPoint: number) {
        this.location = getUniformLocation(gl, program.program, variableName);
        if (im instanceof HTMLImageElement || im instanceof  HTMLCanvasElement) {
            this.texture = createTexture(gl, im);
        } else {
            this.texture = im;
        }
        this.bindPoint = bindPoint;
        this.variableName = variableName;
    }
}


export interface IAttribute {
    location: number;
    value: BufferObject;
    variableName: string;
}


export class Attribute implements IAttribute {

    readonly location: number;
    readonly value: BufferObject;
    readonly variableName: string;
    readonly drawingMode: number;

    constructor(gl: WebGLRenderingContext, program: IProgram, variableName: string, data: number[][], drawingMode: number = gl.TRIANGLES) {
        this.location = getAttributeLocation(gl, program.program, variableName);
        this.value = createFloatBuffer(gl, data, drawingMode);
        this.variableName = variableName;
        this.drawingMode = drawingMode;
    }
}


/**
 * @TODO: this is not yet used anywhere.
 */
export class ElementAttribute implements IAttribute {
    readonly location: number;
    readonly value: BufferObject;
    readonly indices: IndexBufferObject;
    readonly variableName: string;

    constructor(gl: WebGLRenderingContext, program: IProgram, variableName: string, data: number[][], indices: number[][]) {
        this.location = getAttributeLocation(gl, program.program, variableName);
        this.value = createFloatBuffer(gl, data);
        this.indices = createIndexBuffer(gl, indices);
        this.variableName = variableName;
    }
}



function first<T>(arr: T[], condition: (el: T) => boolean): T | null {
    for (const el of arr) {
        if (condition(el)) {
            return el;
        }
    }
    return null;
}


function parseProgram(program: IProgram): [string[], string[], string[]] {
    const attributeRegex = /^\s*attribute (int|float|vec2|vec3|vec4|mat2|mat3|mat4) (\w*);/gm;
    const uniformRegex = /^\s*uniform (int|float|vec2|vec3|vec4|mat2|mat3|mat4) (\w*)(\[\d\])*;/gm;
    const textureRegex = /^\s*uniform sampler2D (\w*);/gm;

    const shaderCode = program.fragmentShaderSource + '\n\n\n' + program.vertexShaderSource;

    const attributeNames = [];
    let attributeMatches;
    while ((attributeMatches = attributeRegex.exec(shaderCode)) !== null) {
        attributeNames.push(attributeMatches[2]);
    }
    const uniformNames = [];
    let uniformMatches;
    while ((uniformMatches = uniformRegex.exec(shaderCode)) !== null) {
        uniformNames.push(uniformMatches[2]);
    }
    const textureNames = [];
    let textureMatches;
    while ((textureMatches = textureRegex.exec(shaderCode)) !== null) {
        textureNames.push(textureMatches[1]);
    }

    return [attributeNames, uniformNames, textureNames];
}


export type RenderMode = 'points' | 'lines' | 'triangles';

interface IShader {
    program: IProgram;
    attributes: IAttribute[];
    uniforms: IUniform[];
    textures: ITexture[];
    bind: (gl: WebGLRenderingContext) => void;
    render: (gl: WebGLRenderingContext, background?: number[], frameBuffer?: FramebufferObject) => void;
    updateAttributeData: (gl: WebGLRenderingContext, variableName: string, newData: number[][]) => void;
    updateUniformData: (gl: WebGLRenderingContext, variableName: string, newData: number[]) => void;
    updateTextureData: (gl: WebGLRenderingContext, variableName: string, newImage: HTMLImageElement | HTMLCanvasElement) => void;
}

export class Shader implements IShader {
    constructor(
        readonly program: IProgram,
        readonly attributes: IAttribute[],
        readonly uniforms: IUniform[],
        readonly textures: ITexture[]
    ) {
        const [attributeNames, uniformNames, textureNames] = parseProgram(program);
        for (const attrName of attributeNames) {
            const found = attributes.filter(a => a.variableName === attrName);
            if (found.length !== 1) {
                throw new Error(`Provided ${found.length} values for shader's attribute ${attrName}.`);
            }
        }
        for (const uniformName of uniformNames) {
            const found = uniforms.filter(a => a.variableName === uniformName);
            if (found.length !== 1) {
                throw new Error(`Provided ${found.length} values for shader's uniform ${uniformName}.`);
            }
        }
        for (const texName of textureNames) {
            const found = textures.filter(a => a.variableName === texName);
            if (found.length !== 1) {
                throw new Error(`Provided ${found.length} values for shader's texture ${texName}.`);
            }
        }
    }

    public bind(gl: WebGLRenderingContext): void {
        bindProgram(gl, this.program.program);
        for (const a of this.attributes) {
            bindBufferToAttribute(gl, a.location, a.value);
        }
        for (const u of this.uniforms) {
            bindValueToUniform(gl, u.location, u.type, u.value);
        }
        for (const t of this.textures) {
            bindTextureToUniform(gl, t.texture.texture, t.bindPoint, t.location);
        }
    }

    public render(gl: WebGLRenderingContext, background?: number[], frameBuffer?: FramebufferObject): void {
        if (!frameBuffer) {
            bindOutputCanvasToFramebuffer(gl);
        } else {
            bindFramebuffer(gl, frameBuffer);
        }
        if (background) {
            clearBackground(gl, background);
        }

        const firstAttribute = this.attributes[0].value;
        drawArray(gl, firstAttribute);
    }


    public updateAttributeData(gl: WebGLRenderingContext, variableName: string, newData: number[][]): void {
        const attribute = first<IAttribute>(this.attributes, el => el.variableName === variableName);
        if (!attribute) {
            throw new Error(`No such attribute ${variableName} to be updated.`);
        }
        updateBufferData(gl, attribute.value, newData);
    }

    public updateUniformData(gl: WebGLRenderingContext, variableName: string, newData: number[]): void {
        const uniform = first<IUniform>(this.uniforms, el => el.variableName === variableName);
        if (!uniform) {
            throw new Error(`No such uniform ${variableName} to be updated.`);
        }
        uniform.value = newData;
    }

    public updateTextureData(gl: WebGLRenderingContext, variableName: string, newImage: HTMLImageElement | HTMLCanvasElement): void {
        const original = first<ITexture>(this.textures, t => t.variableName === variableName);
        if (!original) {
            throw new Error(`No such original ${variableName} to be updated.`);
        }
        const newTextureObject = updateTexture(gl, original.texture, newImage);
        original.texture = newTextureObject;
    }
}


export class Framebuffer {

    readonly fbo: FramebufferObject;

    constructor(gl: WebGLRenderingContext, width: number, height: number) {
        const fb = createFramebuffer(gl);
        const fbTexture = createEmptyTexture(gl, width, height);
        const fbo = bindTextureToFramebuffer(gl, fbTexture, fb);
        this.fbo = fbo;
    }
}




export function renderLoop(fps: number, renderFunction: (tDelta: number) => void): void {

    const tDeltaTarget = 1000 * 1.0 / fps;
    let tDelta = tDeltaTarget;
    let tStart, tNow, tSleep: number;

    const render = () => {
        tStart = window.performance.now();

        renderFunction(tDelta);

        tNow = window.performance.now();
        tDelta = tNow - tStart;
        tSleep = Math.max(tDeltaTarget - tDelta, 0);
        setTimeout(() => {
            requestAnimationFrame(render);
        }, tSleep);

    };

    render();
}







interface IEntity {
    program: IProgram;
    attributes: IAttribute[]; // note that attributes must all have the same number of entries!
    uniforms: IUniform[];
    textures: ITexture[];
    update: (tDelta: number) => void;
}



export class Entity implements IEntity {

    constructor(
        readonly program: IProgram,
        readonly attributes: IAttribute[],
        readonly uniforms: IUniform[],
        readonly textures: ITexture[],
        readonly updateFunction: (tDelta: number, attrs: IAttribute[], unis: IUniform[]) => void) {}

    update(tDelta: number): void {
        this.updateFunction(tDelta, this.attributes, this.uniforms);
    }
}




export class Engine {

    readonly entities: IEntity[] = [];

    constructor() {}

    public renderLoop(gl: WebGLRenderingContext, fps: number): void {
        setup3dScene(gl);

        const tDeltaTarget = 1000 * 1.0 / fps;
        let tStart, tNow: number, tDelta: number, tSleep;
        let currentShader = '';
        const render = () => {
            tStart = window.performance.now();

            // Part 1: allow objects to update their state
            for (const e of this.entities) {
                e.update(tDeltaTarget);
            }

            // Part 2: do the actual rendering work here
            clearBackground(gl, [.7, .7, .7, 1]);
            for (const e of this.entities) {
                if (e.program.id !== currentShader) {
                    bindProgram(gl, e.program.program);
                    currentShader = e.program.id;
                }
                for (const a of e.attributes) {
                    bindBufferToAttribute(gl, a.location, a.value);
                }
                for (const u of e.uniforms) {
                    bindValueToUniform(gl, u.location, u.type, u.value);
                }
                for (const t of e.textures) {
                    bindTextureToUniform(gl, t.texture.texture, t.bindPoint, t.location);
                }
                gl.drawArrays(gl.TRIANGLES, 0, e.attributes[0].value.vectorCount);
            }

            // Part 3: time-management
            tNow = window.performance.now();
            tDelta = tNow - tStart;
            tSleep = Math.max(tDeltaTarget - tDelta, 0);
            setTimeout(() => {
                requestAnimationFrame(render);
            }, tSleep);

        };

        render();
    }

    public addEntity(entity: IEntity): void {
        this.entities.push(entity);
        this.sortEntities();
    }


    private sortEntities(): void {
        this.entities.sort((a: IEntity, b: IEntity) => {
            return (a.program.id > b.program.id) ? 1 : -1;
        });
    }


}
\end{lstlisting}

\subsubsection{Shader gallery}


\subsection{CSS}

\subsubsection{Positioning}
\subsubsection{Events}
\subsubsection{Animations}

