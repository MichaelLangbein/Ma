\section{Programming}


\subsection{Rendering in the browser}
\begin{table}[h]
    \begin{tabular}{@{}llll@{}}
    \toprule
    technology   & use-case                    & programming style &                  & libraries          \\ 
    \midrule       
    canvas/2d    &                             &                   & single-buffering to bitmap. paintbrush-statemachine. &                    \\
    canvas/webgl & 3d, interactive, many items & procedural        & double-buffering to bitmap & threejs, processing \\
    svg + css    & simple 2d graphics          & declarative       & no bitmap, but \inlinecode{<svg>}. has concept of layers, browser-native event-handling. Cannot easily export to image. & d3, raphael              \\ 
    \bottomrule
    \end{tabular}
\end{table}



\subsection{SVG}

SVG consists of 
\begin{itemize}
    \item Objects
    \item Groups. Groups have ...
        \begin{itemize}
            \item a transform
            \item a style
            \item Notably, they don't have a x, y, width or height attribute! If you need those, use SVG's instead
        \end{itemize}
    \item SVG's. 
        \begin{itemize}
            \item x, y width, height
        \end{itemize}
\end{itemize}

Objects have a nice, simple hierarchy.
\begin{itemize}
    \item Objects: each object is a ...
        \begin{itemize}
            \item Path: the most basic object: a Bezier-curve. Each of the following subtypes can be downgraded to a path again in inkscape.
                \begin{itemize}
                    \item rect, star, ellipse, text
                \end{itemize}
        \end{itemize}
    \item Each object has a ... 
        \begin{itemize}
            \item fill, stroke, opacity
        \end{itemize}
\end{itemize}

A bunch of software helps with SVG's:
\begin{itemize}
    \item Inkscape: ideal for creating SVG logos
    \item d3: Data-driven, animated SVG's
    \item Raphael: animated SVG's; imperative, \emph{not} data-driven
\end{itemize}



\subsection{CSS}

\subsubsection{Positioning}
\subsubsection{Events}
\subsubsection{Animations}



\subsection{Webgl}


\subsubsection{Shader gallery}


