\section{Design}

\paragraph{Aesthetics} is not just about making things stand out. It's about encoding information.
\paragraph{Empathy} (UXD) means understanding whom you are working for - how they interpret your choice of color.

\subsection{Layout}

When designing something, just make a note for each of the following points, justifying your choice. This way you can be sure that you did not overlook a useful design-decision.
Common layout-frameworks are bootstrap, clarity, 960, foundation, and neat.io.

\paragraph{Grids} consist of columns, gutter, modules, negative space, margins, and baseline. Some common print-layouts are manuscript, columns and blocks.
\paragraph{Rhythm} states that, no matter the width/font-size/color of any element, its width/text-height/hue is always a multiple of some base-unit. Pick elements/font-sizes/colors according to some scheme, like fibonacci, magic square, etc. Note, however: just like in music, people only recognize patterns when they are repeated. When implementing a pattern, it should appear more than just once on the screen.
\paragraph{Hierarchy}: important elements are bigger and/or have negative space around them. Important items should be close to each other - don't force the users eyes to jump.
\paragraph{Colors} change with time. Currently, the following colorschemes are popular: Material Design, pastels with crisp photography, greyscale with one accent color. Use color only when there is little text on the page. The more text, the more you should cut back on color.

\subsection{UI}

\paragraph{First commandment of UI}: Per default, show as much as neccessary. But allow a user to see his specific points of interest when he wants it.