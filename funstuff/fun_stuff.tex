\section{Fun stuff}

Man kann nicht immer arbeiten. Hier eine Liste von anderen Dingen, die auch die Zeit wert sind:

\begin{itemize}
    \item Malen
    \item Lasertag
    \item Freizeitpark, Trampolinpark
    \item Freibad, Rutschenwelt
    \item Konzerte
    \item Schlittschuhlaufen
    \item Krimi-Dinner
\end{itemize}

\section{Wise stuff}

\begin{itemize}
    \item Feeling stupid is just like being afraid: 
        \begin{itemize}
            \item It's normal. The art lies in getting over it.
            \item If you don't feel it, you're not living to your potential.
        \end{itemize}
        
    \item Perfection is reached not when there is nothing more to add, but when there is nothing more to substract. - Antoine de Saint-Exupéry
    
    \item You can easily stay focused on your task, not by power of will, but by understanding that you don't need to do any of those distracting things, and that they won't make you happy.
    
    \item It is smart to realize when you need a break and how to take it. Knowing that teaches you to listen to your body and to your needs. Take a break early. On the break, don't do anything that strains your mind any further, but relax.
    
    \item Don't take yourself or anyone else too seriously. Remember that deep down you are goof, and anyone else is, too. That is the key to humor and charisma. 
    
    \item a tiny skill perfected beats a lage skill half-baked. If it's too hard, just do one small peace of it, but that really well. Things that you learned only mediocarily will stay bad permanently.  
    
    \item Comradery: 
        \begin{itemize}
            \item Requires organisation: keeping promises and not giving them too easily. Why do you not keep your promises? It feels like you don't live comradery. Strangely, comradery requires quite some level of organisation. You should only give promises when you have a good chance of keeping them. That requires a good organizer. 
            \item Requires social insight: keep track of who acted which way towards you, whom you owe stuff, who has been reliable in the past, who can survive a day without you, who needs you the most right now. Why do you have such a hard time making decissions? When one option is not obviously better, often none really is. For example, in Belgium you had the chióice between visiting Fabi in Brussels or Robin in Antwerp. Under such circumstances you couldn't have made the right decission, because there was none. However, you can keep to guidelines. Comradery would be one.  
        \end{itemize}
    
    \item The only way you can follow through on your carreer is by making every day so that 
        \begin{itemize}
            \item you enjoy the day
                \begin{itemize}
                    \item when you don't like the current project, focus on the part that you do like
                \end{itemize}
            \item you get stuff done
                \begin{itemize}
                    \item by keeping a startup-attitude
                    \item we-all-work-for-each-other
                    \item i-learn-from-everyone-else
                    \item get more efficient every day
                \end{itemize}
        \end{itemize}
\end{itemize}