\section{Angular}


\subsection{Elements}

\begin{itemize}
    \item \emph{Element}: Wrapper around \inlinecode{nativeElement}. Wrapped so as to abstract away differences between native elements in browser and on mobile.
    \item \emph{View}: Smallest group of elements that are created and destroyed together.
        \begin{itemize}
            \item HostView: A view created by a component
            \item EmbeddedView: A view created by a template
        \end{itemize}
    \item \emph{Component}
    \item \emph{Directive}: An attribute that may be appended to a component. Like a component, but without an html-file.
    \item \emph{Template}: Angular-html-files are usually static. The only dynamics that are allowed are via subcomponents or templates \inlinecode{<ng-template>}. Templates are really like simple, inline html-only subcomponents. They are even more like inline-functions in that they have access to their parent-component's template-variables.
    \item \emph{ViewContainer}: removing html-elements from the DOM is dangerous, because angular will not be aware of that removal - it will keep the node in memory and continue running change detection. So how do you remove elements? That's what ViewContainers are for: wrapping a node in a ViewContainer provides us with an API to add or remove components or templates.
\end{itemize}

\subsection{Live cycle}


\subsection{Change detection}
Change detection goes top down: if a user-interaction happened on component X, then the whole tree starting at app-root is updated.
If component X has \inlinecode{changeDetectionStrategy: OnPush}, then X and its children are shielded from updates from outside of X's subtree. But events originating from X or its children still traverse the \emph{whole} tree from app-root down.

\subsection{Service availability}
Services are by default provided by app-root. The instance is used by every child component - except if that child, lets call it X, has its own provided instance of the service. Then all children of X use X's instance of the service.