\section{Maps}

A lot of this information comes from here: \href{http://cite.opengeospatial.org/pub/cite/files/edu/wmts/text/operations.html}{cite.opengeospatial.org}
The map-stack is certainly one of the most demanding stacks of technology that is in widespread use today, and clearly a field where a developer can sharpen his profile.

\subsection{Available software} 

There are so many non-orthogonal tools out there that an overview can't do harm.

\begin{itemize}
    \item Data-storage in db
        \begin{itemize}
            \item Any PostgeSQL binding
            \item Geodjango, for a Website-ORM.
        \end{itemize}
    \item Map-creation
        \begin{itemize}
            \item simple numpy raster data to geo-tiffs: gdal + basemap (deprecated) or cartopy. There is very good documentation here: \href{https://annefou.github.io/metos\_python/04-plotting/}{annefou.github.io}
            \item QGis
            \item MapServer (c)
            \item GeoServer (java)
            \item Mapnik (c, python)
            \item TileMill (python)
        \end{itemize}
    \item Serving data
        \begin{itemize}
	    \item QGis Server
            \item MapServer (c)
            \item GeoServer (java)
            \item Mapnik (c, python)
            \item FeatureServer (python): a WFS 
        \end{itemize}
    \item Displaying data
        \begin{itemize}
            \item Folium: Builds websites based on Leaflet.js, but with a python api.
            \item Geodjango with django-leaflet
            \item OpenLayers
        \end{itemize}
\end{itemize}




\subsection{Server}
A popular choice for serverside software is MapServer (C, CGI and PHP) or GeoServer (Java). Servers locally create GeoTiff files: high resolution images that are geo-referenced. They are then converted to png's and exposed via a HTTP-based API. We need to know about two important implementations. Both of them can be called via SOAP or REST-Parameters (KVP, for key-value-pairs, in geo-lingo). We'll only describe the GET-Parameters here. 

\begin{itemize}
    \item WMS: delivers custom made images for a given map-square. 
    \begin{itemize}
        \item Version 1.1.1
            \begin{itemize}
                \item GetCapabilites: \inlinecode{?service=wms&version=1.1.1&request=GetCapabilites}
                \item GetMap \inlinecode{?service=wms&version=1.1.1&request=GetMap&src=EPSG:31468&format=image/png&BBOX=447,532,234,124&layers=l1,l2&widht=1000&height=1000}
                \item GetFeatureInfo
            \end{itemize}
        \item Version 1.3
            \begin{itemize}
                \item BBOX: andere Achsenreihenfolge! Von vielen clients noch immer nicht richtig verarbeitet. 
            \end{itemize}
    \end{itemize}
    
    \item WMTS: has a pre-made pyramid of images. Clients pass an index for the pyramid instead of the coordinates of the selection. 
    \begin{itemize}
        \item Version 1.0.0.
            \begin{itemize}
                \item GetCapabilities \inlinecode{?service=wmts&version=1.0.0&request=GetCapabilites}
                \item GetTile \inlinecode{?service=wmts&version=1.0.0&request=GetTile&layer=l1&format=image/png&TileMatrixSet=default28m&TileMatrix=1&TileRow=1&TileCol=1}
                \item GetFeatureInfo
            \end{itemize}
    \end{itemize}

    \item WFS: instead of images, a json-representation of features is passed to the client. 
        \begin{itemize}
            \item 
        \end{itemize}
\end{itemize}

Usually, service providers expose \emph{multiple} urls for you to query, so that you can fetch images from several servers in parallel. 

\subsection{Client}
A popular javascript mapping library is openlayers. amongst other things, it facilitates the usage of wms and wmts. Note that openlayers has \emph{two} types of clients for wms: \inlinecode{ImageWms} and \inlinecode{TiledWMS}.

\begin{lstlisting}
require('ol/ol.css');
var Proj         = require('ol/proj').default;
var Extent       = require('ol/extent').default;
var Map          = require('ol/map').default;
var ImageLayer   = require('ol/layer/image').default;
var TileLayer    = require('ol/layer/tile').default;
var ImageWms     = require('ol/source/imagewms').default;
var Wmts         = require('ol/source/wmts').default;
var WmtsTileGrid = require('ol/tilegrid/wmts').default;
var View         = require('ol/view').default;



var projection = Proj.get('EPSG:3857');
var projectionExtent = projection.getExtent();


// Step 1: including a WMS

var wmsLayer = new ImageLayer({
    extent: projectionExtent,
    source: new ImageWms({
        url: "https://ahocevar.com/geoserver/wms",
        params: {"LAYERS": "topp:states"}
    }),
});


// Step 2: including a WMTS

var size = Extent.getWidth(projectionExtent) / 256;
var resolutions = new Array(14);
var matrixIds = new Array(14);
for (var z = 0; z < 14; ++z) {
    resolutions[z] = size / Math.pow(2, z);
    matrixIds[z] = z;
}

var wmtsLayer = new TileLayer({
    source: new Wmts({
        url: 'https://services.arcgisonline.com/arcgis/rest/services/Demographics/USA_Population_Density/MapServer/WMTS/',
        layer: '0',
        matrixSet: 'EPSG:3857',
        style:'default',
        tileGrid: new WmtsTileGrid({
            origin: Extent.getTopLeft(projectionExtent),
            resolutions: resolutions,
            matrixIds: matrixIds
        })
    })
});


// Step 3: display

var layers = [
    wmsLayer,
    wmtsLayer
];


var view = new View({
    center: [0, 0],
    zoom: 0
});


var map = new Map({
  target: 'map',
  layers: layers,
  view: view
});

\end{lstlisting}
