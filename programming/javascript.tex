\section{Javascript}



\subsection{Prototype inheritance}

\begin{lstlisting}
var alien = {
  kind: 'alien'
}

var person = {
  kind: 'person'
}

var zack = {};

// assign alien as the prototype of zack
zack.__proto__ = alien

console.log(zack.kind); //=> alien

// assign person as the prototype of zack
zack.__proto__ = person

console.log(zack.kind); //=> person   
\end{lstlisting}

Prototype lookups are dynamic. You can add properties to the prototype of an object at any time, the prototype chain lookup will find the new property as expected.

\begin{lstlisting}
var person = {}

var zack = {}
zack.__proto__ = person

// zack doesn't respond to kind at this point
console.log(zack.kind); //=> undefined

// let's add kind to person
person.kind = 'person'

// now zack responds to kind
// because it finds 'kind' in person
console.log(zack.kind); //=> person
\end{lstlisting}

\subsection{Constructor functions}

Honestly, I dislike constructor functions. 
Constructor functions are factories of objects. They have a \inlinecode{prototype}  property. But watch out! 
\inlinecode{ Robot.__proto__} is the prototype of the function \inlinecode{Robot}, whereas \inlinecode{Robot.prototype}  is the prototype of the object that the function generates.
In fact, every function in javascript has a \inlinecode{prototype}  property, because there is no way of knowing which function is going to be used as a constructor function. 

\begin{lstlisting}
var Robot = function () {
    this.sayHi = function () {
        console.log("hi!");
    };
};
var r2d2 = new Robot();

Robot.__proto__ // => 'native code'
Robot.prototype // => the prototype of r2d2
\end{lstlisting}

\subsection{Module Pattern}
A module is created with a self-executing closure: 

\begin{lstlisting}
// buz.js
var buz = (function ($) {
    var buz;
    
    buz.sayHi = function () {
        console.log("hi!");
    };
    
    return buz;
} ($));
\end{lstlisting}

Ultimately, this can be a little hard to read. There are tools like browserify, however, that let you write your sourcecode locally like you would in node, and then bundle all your requires together into one single script. 









\subsection{Node and NPM}
Node is a webserver that you can program in javascript, npm is a javascript package manager. 

\paragraph{Serving files with node}

\paragraph{Creating  projects with npm}

Initiate a project: \inlinecode{npm init}

Npm has already done most stuff in the \verb package.json  for you. But you still need to add the dependencies and the start-script like for example so: 

\begin{lstlisting}
{
  "name": "electron-quick-start",
  "version": "1.0.0",
  "description": "A minimal Electron application",
  "main": "main.js",
  "scripts": {
    "start": "electron ."
  },
  "repository": "https://github.com/electron/electron-quick-start",
  "author": "GitHub",
  "dependencies": {
    "electron": "^1.6.2"
  }
}
\end{lstlisting}


Fetch dependencies using \inlinecode{npm install}. You can add further dependencies to your \verb package.json  by calling \inlinecode{npm install --save-dev gulp}, where \inlinecode{--save-dev} means that we only want to use the \inlinecode{gulp} package during development, but not in the final, deployed product.
Run the app with \inlinecode{npm start} (Which in turn only just calls "electron .").

Uninstall packages with \inlinecode{npm uninstall <package-name>}


\paragraph{Modules in npm}

A module is created with a pretty simple syntax. Contrary to the closure-module-pattern we'd use in web apps, in node there is no need for a closure. The only thing that gets exported is what we put in $module.exports$:

\begin{lstlisting}
var myMod = {
    sayHi : sayHi
};
var sayHi = function () {
    console.log("hi!");
};
var privateMethod = function () {
    console.log("this method won't be visible outside of this script");
};
module.exports = myMod;
\end{lstlisting}

Once some value for $module.exports$ has been set, it can be imported as such: 

\begin{lstlisting}
    // app.js
    var myMod = require('./myMod.js');
    myMod.sayHi();
\end{lstlisting}




\paragraph{Buildautomation}

If you have a really complex build-process, you might want to use something like gulp. However, for a simple setup, npm's \verb scripts   are more than enough. 

\begin{lstlisting}
{
  "name": "fstsktch",
  "main": "index.js",
  "scripts": {
    "devel": "watchify src/js/sketch.js -o build/main.js",
    "deploy": "npm run brwfy && npm run uglfy",

    "brwfy": "browserify src/js/sketch.js -o build/deleteMe/bundle.js",
    "uglfy": "uglify -s build/deleteMe/bundle.js -o build/main.js"
  },
  "devDependencies": {
    "browser-sync": "^2.23.6",
    "browserify": "^16.0.0",
    "uglify": "^0.1.5",
    "watchify": "^3.10.0"
  },
  "dependencies": {
    "p5": "^0.6.0"
  }
}
\end{lstlisting}

Your code will be automatically updated with \inlinecode{npm run devel}. It will be minified and deployed with \inlinecode{npm run deploy}.


\paragraph{Browser} Traditionally, chrome and firefox are good choices for development, because they offer a nice developer console. In firefox, it can be tricky to avoid the js cache. Under settings, \emph{disable} the field "HTTP cache bei offenem Werkzeugkasten deaktivieren". After doing so, the combination \inlinecode{ctrl - shift - r} will reload all js. 




\subsection{Var, let and const}
\begin{itemize}
    \item use \inlinecode{var} for a variable scoped inside this function. 
    \item use \inlinecode{let} for a variable scoped inside this curly braces (this often a smaller scope than this function. Consider for example for-loops. Var will still exist after a for-loop, while let only exists inside it.) 
    \item use \inlinecode{const} for a constant, that is, a variable that gets a value once and cannot  thereafter be reassigned. (Works as expected when using primitives. But notice that in \inlinecode{const dog = {age:3};} we can still change the age-property.)
\end{itemize}


\subsection{Events versus Promises}
Javascript doesn't do multithreading, but it does do asynchronous execution. Both events and promises are a tool that is meant to help you deal with asynchronous code. 

Events are great for things that can happen multiple times on the same object—keyup, touchstart etc. Here is an example of how we would use events: 
\begin{lstlisting}
var img1 = document.querySelector('.img-1');

img1.addEventListener('load', function() {
  // woo yey image loaded
});

img1.addEventListener('error', function() {
  // argh everything's broken
});

\end{lstlisting}

