\section{Emacs}

\subsection{Basics}

\inlinecode{c-g}: cancel current action.


\paragraph{help} show a lot of info on the right sidebar: 
\begin{itemize}
\item \inlinecode{C-h k key-binding}: Describe the function bound to the key binding. To get this to work, you actually perform the key sequence after typing C-h k.
\item \inlinecode{C-h f}: Describe function. 
\end{itemize}

\paragraph{buffers} are the content of files:
\begin{itemize}
    \item \inlinecode{c-x b}: create new buffer.
    \item \inlinecode{c-x k}: kill buffer.
\end{itemize}


\paragraph{files} are what we usually work with instead of buffers (technically, a buffer is just the content of a file, or some content not yet saved to a file):
\begin{itemize}
    \item \inlinecode{c-x c-f}: opens a file
    \item \inlinecode{c-x c-s}: save file
    \item \inlinecode{c-x c-c}: close file
\end{itemize}


\paragraph{Modes} Emacs has a dedicated mode for almost everything. There is a python mode, a git mode, a compare mode, and much more:
\begin{itemize}
    \item Chosing a mode: \inlinecode{M-x}
    \item Refreshing package-list: \inlinecode{M-x package-refresh}
    \item \inlinecode{M-x package-list}
    \item \inlinecode{M-x package-install}
    \item \inlinecode{M-x replace-string}
    \item \inlinecode{M-x ediff}: compare mode
    \item \inlinecode{M-x dired}: file manager
\end{itemize}
You have one major mode (decided on by the file type) and many minor modes. A minor mode is toggled on or off with the same command. 


\paragraph{moving}:
\begin{itemize}
    \item \inlinecode{C-a} 	Move to beginning of line.
    \item \inlinecode{M-m} 	Move to first non-whitespace character on the line.
    \item \inlinecode{C-e} 	Move to end of line.
    \item \inlinecode{C-f} 	Move forward one character.
    \item \inlinecode{C-b} 	Move backward one character.
    \item \inlinecode{M-f} 	Move forward one word (I use this a lot).
    \item \inlinecode{M-b} 	Move backward one word (I use this a lot, too).
    \item \inlinecode{C-s} 	Regex search for text in current buffer and move to it. Press C-s again to move to next match.
    \item \inlinecode{C-r} 	Same as C-s, but search in reverse.
    \item \inlinecode{M-<} 	Move to beginning of buffer.
    \item \inlinecode{M->} 	Move to end of buffer.
    \item \inlinecode{M-g g} Go to line. 
\end{itemize}

\paragraph{Selecting} (or, rather, creating regions):
\begin{itemize}
    \item \inlinecode{C-spc}  start selecting
    \item 
\end{itemize}

\paragraph{copy-pasting}:
\begin{itemize}
    \item \inlinecode{c-/}: undo
    \item \inlinecode{C-w}:	Kill region.
    \item \inlinecode{M-w}: 	Copy region to kill ring.
    \item \inlinecode{C-y}: 	Yank.
    \item \inlinecode{M-y}: 	Cycle through kill ring after yanking.
    \item \inlinecode{M-d}: 	Kill word.
    \item \inlinecode{C-k }:	Kill line. 
\end{itemize}

\paragraph{spacing and expanding}:
\begin{itemize}
    \item \inlinecode{Tab}: 	Indent line.
    \item \inlinecode{C-j}: 	New line and indent, equivalent to enter followed by tab.
    \item \inlinecode{M-/}: 	Hippie expand; cycles through possible expansions of the text before point.
    \item \inlinecode{M-\\}: 	Delete all spaces and tabs around point. (I use this one a lot.) 
\end{itemize}

\paragraph{windows and frames}:
\begin{itemize}
    \item \inlinecode{C-x o}: 	Switch cursor to another window. Try this now to switch between your Clojure file and the REPL.
    \item \inlinecode{C-x 1}: 	Delete all other windows, leaving only the current window in the frame. This doesn’t close your buffers, and it won’t cause you to lose any work.
    \item \inlinecode{C-x 2}: 	Split frame above and below.
    \item \inlinecode{C-x 3}: 	Split frame side by side.
    \item \inlinecode{C-x 0}: 	Delete current window. 
\end{itemize}


\subsection{Clojure}



\subsection{Writing custom commands}
This is a really simple two step process: 

In your \inlinecode{.emacs.d/init.el} file, create a function
\begin{lstlisting}[language=lisp]
(defun insert-line-before () 
    (interactive)           ;; <--- a function call that tells emacs that this is a function the user can call
    (save-excurstion 
        (move-beginning-of-line 1)
        (newline)))    ;; <--- the actual function body
\end{lstlisting}

Now the function call already be executed using \inlinecode{M-x insert-line-before}.

But to make the function available via a shortcut, you also need to do: 
\begin{lstlisting}[language=lisp]
(global-set-key 
    (kbd "C-S-i")
    'insert-line-before)
\end{lstlisting}


