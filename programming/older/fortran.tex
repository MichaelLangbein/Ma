\section{Fortran}

Fortran can be even faster than c, because it can make use of the very strict rules of the language in its compiler. 


\paragraph{The differences to c} are an easy way to get a first impression of how fortran works. 
\begin{itemize}
    \item Case-insensitive
    \item Pointers always have a type. You're not allowed to cast pointers to void or any other type. 
    \item Pointers are not integers. In case of an array, pointers contain information about the array-size. 
    \item You can pass functions as arguments, but not in the form of a pointer. 
    \item Variables are always passed by reference, never by value or by pointer (just like in java).
    \item Functions are not by default allowed to be called recursively. 
\end{itemize}

\subsection{Basic syntax}
\begin{lstlisting}[language=fortran]
program HelloWorld
  ! This is a comment
  write (*,*) "Hello world."
end program
\end{lstlisting}
The parameter \inlinecode{(*,*)} tells the compiler to print the output to the terminal stdout (first star) with default printing format (second star)

Variables must be defined in the beginning of a program/subroutine/function. Typical variables are integer variables, floating point variables, boolean variables, character variables and strings. Using the operator "=", values can be assigned to variables. The following instructive example demonstrates the definition.
\begin{lstlisting}[language=fortran]
program DefineVariables

  integer :: i      ! Integer variable
  real :: d         ! Floating point value
  logical :: b      ! Boolean variable
  character :: c    ! Character

  character(len=10) :: sstring  ! String of length 10

  i = 0
  d = 0.0
  b = .false.       ! Logical has values ".true." or ".false."
  sstring = "Hello"

  write (*,*) i, " ", d, " ", b, " ", sstring
end program
\end{lstlisting}
