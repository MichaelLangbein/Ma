\section{C++}

C++ is the best compromise between raw power and versatility. Fortran is faster in large matrix-operations, but lacks the abstraction of OOP and any multi-media support. Java is even better abstracted and has an unbeatable ecosystem with modules for everything, but slower. 



\subsection{Some weird syntax and new concepts}
Contrary to garbage-collected languages, where you only need to care about who knows about a given object, in cpp you also net to worry about where that object lives. 

\subsubsection{References}
Refences are really just syntactic sugar around pointers. It's still useful to know a few of their basic properties, since they are intended to help you avoid errors in memory management. 

Using pointers, to mutate an object you need to use pointer syntax for any operation on the object. This is known as dereferencing: using \inlinecode{*vpt} and \inlinecode{vpt->}.
\begin{lstlisting}[language=c++]
void foo(Vec* vpt){
    vpt->increment(1);
}

Vec vec = Vec();
foo(&vec);
\end{lstlisting}

Using references we can act as if the actual object and not just a pointer to it was in the function scope.
\begin{lstlisting}[language=c++]
void foo(Vec& v){
    v.increment(1);
}

Vec vec = Vec();
foo(vec);
\end{lstlisting}

Syntax: 
\begin{lstlisting}[language=c++]
int i = 3;
int* i_ptr = &i;
int& i_ref = i;
\end{lstlisting}

There are a few more noteworthy properties of references that are intended to make your life easier:
\begin{itemize}
    \item A pointer may point to nothing\footnote{However, this is very bad practice. Always initialize your free pointers to null using \inlinecode{int* ap = nullptr;}} - you may write \inlinecode{int* nullpt;}. A reference however mustn't do that. This will not compile: \inlinecode{int& nullrf;}.
    \item You can point a pointer to a new object, but a reference allways points to the same object. This will have \inlinecode{ap} point to \inlinecode{b}: 
    \begin{lstlisting}
    int a = 1; 
    int b = 2; 
    int& ar = a; 
    ar = b;
    \end{lstlisting}
    This, however, will overwrite \inlinecode{a}: 
    \begin{lstlisting}
    int a = 1; 
    int b = 2; 
    int& ar = a; 
    ar = b;
    cout << a << endl;  // yields 2 (!)
    \end{lstlisting}
\end{itemize}
\subsubsection{Const keyword}
Sometimes you want immutability; in cpp probably even more than in other languages, for security reasons. That is what the \inlinecode{const} keyword is there for. In most cases, you'll use this keyword when a function is passed some objects that it mustn't alter. Passing the objects as consts guarantees that the function has no sideeffects on the objects. 
\begin{lstlisting}[language=c++]
bool isEqual(const Vec & v1, const Vec & v2) {
    bool eq = false;
    if( v1.x == v2.x && v1.y == v2.y ){
        eq = true;
    }
    return eq;
}
\end{lstlisting}

Of course, we could also have avoided side-effects on v1 and v2 by just passing by value, because then the function would have only altered it's own copies of v1 and v2. But creating those copies is a costly operation, so we're cheeper of by passing by reference and using const to achieve immutability. 

Unfortunately, \inlinecode{const} has some weird syntax rules. The trick is in figuring out whether a value or a pointer to a value is to be constant. The rule is that \inlinecode{const} allways applies to the keyword to it's left. If there is nothing on the left, it applies to the keyword on the right. 
\begin{lstlisting}
const int a = 1;                // a constant integer
const int* a_ptr = nullptr;     // a variable pointer to a constant integer
int const* a_ptr;               // a variable pointer to a constant integer
int * const a_ptr;              // a constant pointer to a variable integer (you will rarely need this)
\end{lstlisting}




\subsection{Object orientation}
Here is a simple example of the class-syntax in cpp.

\begin{lstlisting}[language=c++]
#include <iostream>
#include <string>

using std::string;

class Robot {
  
  private:
    string name;
    int hitpoints;
  
  public:
    Robot(string n, int hp) : name(n), hitpoints(hp){
      std::cout << "Created " << name << " with " << hitpoints << " health" << std::endl;
    }
    
    void hit(Robot* other){
      other->takeHit(10);
    }
    
    void takeHit(int strength){
      hitpoints -= strength;
      std::cout << name << " now has " << hitpoints << " health" << std::endl;
    }
    
    ~Robot(){
      std::cout << name << " destroyed." << std::endl;
    }
};

int main() {
  Robot b("Bender", 100);
  Robot v("Vender", 100);
  b.hit(&v);
  return 0;
}
\end{lstlisting}

From this basic example we can already note a few important points.


\paragraph{Constructors} cpp knows \emph{three} ways of creating objects: 
\begin{itemize}
    \item The default constructor \inlinecode{Robot b("Bender", 100);} is what you should usually make use of. 
 
    \item Construction by copying an existing instance \inlinecode{Robot b("Bender", 100); Robot v = b;} requires a functioning implementation of the copy constructor \inlinecode{Robot(const Robot& otherRobot)}. If this isn't provided, the compiler generates one automatically. The copy constructor is called in three cases: 
    	\begin{itemize}
    		\item When an object is created by copying: \inlinecode{Robot b("Bender", 100); Robot v = b;} or \inlinecode{Robot b("Bender", 100); Robot v(b);}
    		\item When passing an object to a function by value: \inlinecode{void someFunc(Robot r);} . Note, however, that it with big objects it may be better to pass the object by (const) reference: \inlinecode{void someFunc(cont Robot& r);}
    		\item When returning an object from a function by value: \inlinecode{Robot someFunc(); Robot b = sumeFunc();}
    	\end{itemize} . Here, too, it is usually more efficient to use references instead of copying by value. 
    
    \item Construction by assignment operator \inlinecode{Robot b("Bender", 100); Robot v("Vender", 90); v = b;} requires a functioning implementation of the assignment operator \inlinecode{Robot& operator=(const Robot& otherRobot)\{ ... return *this;\}}. If this isn't provided, the compiler generates one automatically. 
\end{itemize}

Here an example to clarify the difference between assignment and copy-constructor:

\begin{lstlisting}[language=c++]
#include<iostream>
#include<stdio.h>
 
using namespace std;
 
class Test{
  public:
    Test() {}
     Test(const Test &t) {
      cout<<"Copy constructor called "<<endl;
     }
     Test& operator = (const Test &t) {
        cout<<"Assignment operator called "<<endl;
	return *this;
     }
};
 
int main() {
  Test t1, t2;
  t2 = t1;
  Test t3 = t1;
  getchar();
  return 0;
}
\end{lstlisting}

\paragraph{Constructors as objects or as pointers} Objects can be constructed as objects or as pointers to objects. 

Construction as object: 
\begin{lstlisting}[language=c++]
Particle p = Particle(20, 30);
p.move();
\end{lstlisting}
Here the object is automatically destroyed once it goes out of scope. Note the missing \inlinecode{new} keyword.

Construction as pointer: 
\begin{lstlisting}[language=c++]
Particle* p = new Particle(20, 30);
p->move();
delete p;
\end{lstlisting}
Here, the object survives until you manually call \inlinecode{delete p}.

Whenever possible, you should prefer the first method. However, there are some situations when it makes sense to use pointers: 

\begin{itemize}
    \item when the object is returned from a function (or any other situation where the object needs to outlive its scope)
    \item When you're passing the object to a function and you want that function to change the object and not just a copy of the object. 
\end{itemize}



\paragraph{Initialisation} Note also this important difference. In java, when you write \inlinecode{Perticle p;} this creates an uninitialized reference. In c++ however, this same command already does the initialisation by calling the default constructor.



\subsubsection{Operator overloading}
You can override any operator, as long as the operation makes use of at least one custom datatype. 

\begin{lstlisting}[language=c++]
#include <iostream>
#include <string>

using std::string;

class Vector {
  public:
    int x;
    int y;
    int z;
    Vector(int _x, int _y, int _z) : x(_x), y(_y), z(_z) {
      std::cout << "Created: [" << x << "," << y << "," << z << "]" << std::endl;
    }
    ~Vector(){
      std::cout << "Going out of scope: [" << x << "," << y << "," << z << "]" << std::endl;
    }
};


Vector operator+(Vector const & v1, Vector const & v2) {
  int x = v1.x + v2.x;
  int y = v1.y + v2.y;
  int z = v1.z + v2.z;
  Vector v3 = Vector(x, y, z);
  return v3;
}

int main() {
  
  Vector a = Vector(1, 2, 3);
  Vector b = Vector(2, 3, 4);
  Vector c = a + b;
  return 0;
}
\end{lstlisting}

\paragraph{A very special case: assignment operator} The assignment operator is a very special case. It is only allowed to be defined inside a class. But it is very useful to get an insight into what happens when you do an assignment in c++.
example with overwriting an object

From this we learn a very important lesson about the assignment operator. When one object is assigned to another, a copy of the actual values is made. In Java, copying an object variable merely establishes a second reference to the object. Copying a C++ object is just like calling clone in Java. Modifying the copy does not change the original.
\begin{lstlisting}[language=c++]
Point q = p; /* copies p into q */
q.move(1, 1); /* moves q but not p */
\end{lstlisting}


\paragraph{Assignment by copying versus assignment by moving} The last paragraph was not absolutely honest with you: you can force cpp to move an object instead of copying it when doing an assignment. 
ex with std::move



\subsubsection{Rule of three}
The rule of three states that if you write a custom constructor or destructor, you're also going to need a custom copy-constructor and a custom assignment-operator. The rule basically comes down to this: if you do any memory management during construction, you will also do memory management during copying and during destruction. The compiler cannot guess that memory management for you, so you need to specify it yourself. 

\begin{lstlisting}[language=c++]
class Ball {
    private:
        int xpos;
		int ypos;
		int xvel;
		int yvel;

    public:
        Ball(int x, int y) : xpos(x), ypos(y) {   // Standard constructor
        	std::cout << "ball created" << std::endl;
        }
        		
        ~Ball() {
        	std::cout << "ball deleted" << std::endl;
        }
        
        Ball(const Ball&) = delete;    // Copy constructor (note that argument has no name when using delete)
		Ball& operator=(const Ball&) = delete;     // Assignment operator (note that argument has no name when using delete)
        
        void move() {
        	xpos += xvel;
        	ypos += yvel;
        }
        		
        void push(int dvx, int dvy) {
        	xvel += dvx;
        	yvel += dvy;
        }
};
\end{lstlisting}



\subsubsection{Smart pointers}
With a \inlinecode{unique_ptr} you can make sure that only one object, the owner of the pointer, can access the value behind the pointer. Consider this epic tale of a king and his magic sword: 
\begin{lstlisting}[language=c++]
#include <iostream>
#include <string>
#include <memory>

using std::string;

class Weapon {
  private:
    string name;
  
  public:
    Weapon(string n = "rusty sword") : name(n) {
    
    }
    
    ~Weapon(){
      std::cout << name << " destroyed" << std::endl;
    }
    
    string getName() {
      return name;
    }
};

class Hero {
  private:
    string name;
    Weapon* weapon_ptr;
  
  public:
    Hero(string n) : name(n), weapon_ptr(nullptr) {}
    
    ~Hero(){
      if(weapon_ptr != nullptr){ 
        std::cout << name << " now destroying " << weapon_ptr->getName() << std::endl;
        delete weapon_ptr;
      }
      std::cout << name << " destroyed." << std::endl;
    }
    
    void pickUpWeapon(Weapon* w) {
      if(weapon_ptr != nullptr) delete weapon_ptr;
      weapon_ptr = w;
    }
    
    string describe() {
      if(weapon_ptr != nullptr){
        return name + " now swings " + weapon_ptr->getName();
      } else {
        return name + " now swings his bare hands";
      }
    }
};

Weapon* blacksmithForge(string name){
  Weapon* wp = new Weapon(name);
  return wp;
}


int main() {
  
  Hero arthur = Hero("Arthur");
  Hero mordred = Hero("Mordred");
  Weapon* excalibur = blacksmithForge("Excalibur");
  
  std::cout << arthur.describe() << std::endl;
  arthur.pickUpWeapon(excalibur);
  std::cout << arthur.describe() << std::endl;
  mordred.pickUpWeapon(excalibur);
  std::cout << mordred.describe() << std::endl;
  std::cout << "Everything going out of scope." << std::endl;
  
  return 0;
}
\end{lstlisting}

Note how Mordred has snatched away Arthurs sword. When mordred dies, he takes Excalibur with him into the grave. When Arthur dies, he tries to do the same, but has to find that the sword he thought he had was no longer there. We can fix this problem by using \inlinecode{unique_ptr}'s:

\begin{lstlisting}[language=c++]
#include <iostream>
#include <string>
#include <memory>

using std::string;

class Weapon {
  private:
    string name;
  
  public:
    Weapon(string n = "rusty sword") : name(n) {};
    
    string getName() {
      return name;
    }
};

class Hero {
  private:
    string name;
    std::unique_ptr<Weapon> weapon_ptr;
  
  public:
    Hero(string n) : name(n), weapon_ptr(new Weapon(n + "'s bare hands")) {}
    
    void pickUpWeapon(std::unique_ptr<Weapon> w) {
      if(!w){
        std::cout << "Nothing to pick up." << std::endl;
        return;
      }
      weapon_ptr = std::move(w);
    }
    
    string describe() {
        return name + " now swings " + weapon_ptr->getName();
    }
};

std::unique_ptr<Weapon> blacksmithForge(string name){
  std::unique_ptr<Weapon> wp(new Weapon(name));
  return wp;
}


int main() {
  
  Hero arthur = Hero("Arthur");
  Hero mordred = Hero("Mordred");
  std::unique_ptr<Weapon> excalibur = blacksmithForge("Excalibur");
  
  std::cout << arthur.describe() << std::endl;
  arthur.pickUpWeapon(std::move(excalibur));
  std::cout << arthur.describe() << std::endl;
  
  std::cout << "Mordred now tries to snatch Excalibur." << std::endl;
  mordred.pickUpWeapon(std::move(excalibur));
  std::cout << mordred.describe() << std::endl;
  

  std::cout << "Everything going out of scope." << std::endl;
  
  return 0;
}
\end{lstlisting}




\subsubsection{Ownership: unique, shared and weak pointers}
Let us explain in a bit more detail what a smart pointer is. Really, it i just a class encapsulating a raw pointer. The idea is that by wrapping a pointer in a class, we can have the class object be allocated on the stack and have it free the memory behind the pointer when the object is destroyed. This way, you don't have to manually delete a pointer.

\begin{lstlisting}[language=c++]
class SmartPointer{
    private:
        T* ptr;
    public:
        SmartPointer(T* p){
            ptr = p;
        }
        ~SmartPointer(){
            free(ptr);
        }
}
\end{lstlisting}

In this example, we allocate Excalibur on the heap. But because we're using smart pointers, we never have to clean up memory manually:
\begin{lstlisting}[language=c++]
#include <iostream>
#include <memory>

using std::string;

class Weapon{
  private:
    string name;
  public:
    Weapon(string n) : name(n){}
    string getName(){
      return name;
    }
    ~Weapon(){
      std::cout << name << " is being destroyed" << std::endl;
    }
};

class Hero{
  private:
    string name;
    std::unique_ptr<Weapon> weapon;
  public:
    Hero(string n) : name(n), weapon(new Weapon("rusty sword")) {}
    void pickWeapon(std::unique_ptr<Weapon> w){
      std::cout << name << " picked up " << w->getName() << std::endl;
      weapon = std::move(w);
      std::cout << weapon->getName()  << " is now on " << name << std::endl;
    }
    ~Hero(){
      std::cout << name << " is being destroyed" << std::endl;
    }
};

int main() {
  
  Hero arthur("Arthur");
  std::unique_ptr<Weapon> w(new Weapon("Excalibur"));
  arthur.pickWeapon(std::move(w));
  
  return 0;
}
\end{lstlisting}

Note how we used \inlinecode{move(w)} to overwrite \inlinecode{weapon}. What happens here is this: In the main-scope, the \inlinecode{w}'s internal pointer to Excalibur is replaced by a \inlinecode{nullptr}. Any further attempt of accessing \inlinecode{w} will cause a runtime-error. Inside the \inlinecode{pickWeapon} method, the old smartpointer to the rusty sword is destroyed, causing the sword itself to be free'd.  
We could'nt have used \inlinecode{weapon = w;}; that would have caused an error (because \inlinecode{unique_ptr}'s copy-assignment method is delted). This is deliberate: we don't want copies of a unique pointer to exist. We want a unique pointer to only ever exist in one place (in this case: first in the \inlinecode{main} function, later in the \inlinecode{arthur} instance), so that the memory behind the pointer can only get freed once (in this case: when \inlinecode{arthur} is destroyed).

Generally, we can proclaim the following rules:
\begin{itemize}
    \item When you're given a reference, someone else will clean up the original. 
    \item When you're given a unique pointer, you are now the sole owner of the smartpointer. If you move the pointer, it will go out of scope once you do; if you don't, it will go out of scope once your method ends. Passing a unique pointer to a method that doesn't move the pointer means destroying the object! To pass a unique pointer to a method without it being destroyed or moved, don't pass the unique pointer, pass a \inlinecode{unique_ptr<T> const &}.
    \item When you're given a shared pointer, ... . Shared pointers use reference counting just like the java garbage collector.  
\end{itemize}
There are a lot more useful tips \hyperlink{https://herbsutter.com/2013/06/05/gotw-91-solution-smart-pointer-parameters/}{here} and \hyperlink{https://stackoverflow.com/questions/8114276/how-do-i-pass-a-unique-ptr-argument-to-a-constructor-or-a-function}{here}.


\paragraph{Using ownership wisely} Based on this, we can define different ownership strategies depending on whether an object is a value-object or an id-object. 


\subsection{Templates}
Templates are like java generics. 

\subsubsection{Function template}

\begin{lstlisting}[language=c++]
#include <iostream>
#include <string>

using namespace std;

template <typename T>
inline T const& Max (T const& a, T const& b) { 
   return a < b ? b:a; 
}

int main () {
   int i = 39;
   int j = 20;
   cout << "Max(i, j): " << Max(i, j) << endl; 

   double f1 = 13.5; 
   double f2 = 20.7; 
   cout << "Max(f1, f2): " << Max(f1, f2) << endl; 

   string s1 = "Hello"; 
   string s2 = "World"; 
   cout << "Max(s1, s2): " << Max(s1, s2) << endl; 

   return 0;
}
\end{lstlisting}


\subsubsection{Class template}


\subsection{Threading} 
Comared to c's posix threads, c++ has somewhat simplified it's core threading library. 
\begin{lstlisting}[language=c++]
#include <iostream>
#include <thread>

void threadFunc(int i) {
  for(int j = 0; j < 100; j++){
    std::cout << "hi! this is operation " << j <<" from thread nr " << i << std::endl;
  }
}

int main(){
  std::thread threads[10];
  
  for(int i = 0; i<10; i++){
    threads[i] = std::thread(threadFunc, i);
  }
  
  for(int i = 0; i<10; i++){
    threads[i].join();
  }
  
  return 0;
}
\end{lstlisting}
You can avoid some of the above problem using barriers in your code (\inlinecode{std::mutex}) which will let you synchronize the way a group of threads share a resource.




\subsection{CMake}
CMake input files are written in the “CMake Language” in source files named CMakeLists.txt or ending in a .cmake file name extension.

CMake Language source files in a project are organized into:
\begin{itemize}
    \item \emph{Directories} (CMakeLists.txt). When CMake processes a project source tree, the entry point is a source file called CMakeLists.txt in the top-level source directory. This file may contain the entire build specification or use the \inlinecode{add_subdirectory()} command to add subdirectories to the build. Each subdirectory added by the command must also contain a CMakeLists.txt file as the entry point to that directory. For each source directory whose CMakeLists.txt file is processed CMake generates a corresponding directory in the build tree to act as the default working and output directory.
    \item \emph{Scripts} (script.cmake). An individual script.cmake source file may be processed in script mode by using the cmake(1) command-line tool with the -P option. Script mode simply runs the commands in the given CMake Language source file and does not generate a build system. It does not allow CMake commands that define build targets or actions.
    \item \emph{Modules} (module.cmake). CMake Language code in either Directories or Scripts may use the include() command to load a module.cmake source file in the scope of the including context. See the cmake-modules(7) manual page for documentation of modules included with the CMake distribution. Project source trees may also provide their own modules and specify their location(s) in the \inlinecode{CMAKE_MODULE_PATH} variable.
\end{itemize}

\subsubsection{Variables}
Variables are the basic unit of storage in the CMake Language. Their values are always of string type, though some commands may interpret the strings as values of other types. The \inlinecode{set()} and \inlinecode{unset()} commands explicitly set or unset a variable, but other commands have semantics that modify variables as well. Variable names are case-sensitive and may consist of almost any text, but we recommend sticking to names consisting only of alphanumeric characters plus \inlinecode{_} and \inlinecode{-}.

Although all values in CMake are stored as strings, a string may be treated as a list in certain contexts, such as during evaluation of an Unquoted Argument. In such contexts, a string is divided into list elements by splitting on \inlinecode{;} characters not following an unequal number of \inlinecode{[} and \inlinecode{]} characters and not immediately preceded by a \inlinecode{\\}. The sequence \inlinecode{\\;} does not divide a value but is replaced by \inlinecode{;} in the resulting element.

A list of elements is represented as a string by concatenating the elements separated by \inlinecode{;}. For example, the \inlinecode{set()} command stores multiple values into the destination variable as a list:
\begin{lstlisting}
set(srcs a.c b.c c.c) # sets "srcs" to "a.c;b.c;c.c"
\end{lstlisting}
Lists are meant for simple use cases such as a list of source files and should not be used for complex data processing tasks. Most commands that construct lists do not escape \inlinecode{;} characters in list elements, thus flattening nested lists:
\begin{lstlisting}
set(x a "b;c") # sets "x" to "a;b;c", not "a;b\;c"
\end{lstlisting}


A variable reference has the form \inlinecode{\$\{variable_name\}} and is evaluated inside a Quoted Argument or an Unquoted Argument. A variable reference is replaced by the value of the variable, or by the empty string if the variable is not set. Variable references can nest and are evaluated from the inside out, e.g. \inlinecode{\$\{outer_\$\{inner_variable\}_variable\}}.

An environment variable reference has the form \inlinecode{\$ENV\{VAR\}} and is evaluated in the same contexts as a normal variable reference.

\subsubsection{Logic}
\begin{itemize}
    \item Conditional Blocks: The \inlinecode{if()/elseif()/else()/endif()} commands delimit code blocks to be executed conditionally.
    \item Loops: The \inlinecode{foreach()/endforeach()} and \inlinecode{while()/endwhile()} commands delimit code blocks to be executed in a loop. The \inlinecode{break()} command may be used inside such blocks to terminate the loop early.
    \item Command Definitions: The \inlinecode{macro()/endmacro()}, and \inlinecode{function()/endfunction()} commands delimit code blocks to be recorded for later invocation as commands.
\end{itemize}


\subsubsection{Command arguments}
\begin{itemize}
    \item bracket argument:\begin{lstlisting}
message([=[
This is the first line in a bracket argument with bracket length 1.
No \-escape sequences or ${variable} references are evaluated.
This is always one argument even though it contains a ; character.
The text does not end on a closing bracket of length 0 like ]].
It does end in a closing bracket of length 1.
]=])
\end{lstlisting}

    \item quoted argument:\begin{lstlisting}
message("This is a quoted argument containing multiple lines.
This is always one argument even though it contains a ; character.
Both \\-escape sequences and ${variable} references are evaluated.
The text does not end on an escaped double-quote like \".
It does end in an unescaped double quote.
")
\end{lstlisting}

    \item unquoted argument: \begin{lstlisting}
foreach(arg
    NoSpace
    Escaped\ Space
    This;Divides;Into;Five;Arguments
    Escaped\;Semicolon
    )
  message("${arg}")
endforeach()
\end{lstlisting}
\end{itemize}


\subsubsection{Important predefined variables}
There is a list of CMake’s global varibales. You should know them.
\begin{itemize}
    \item \inlinecode{CMAKE_BINARY_DIR}: if you are building in-source, this is the same as \inlinecode{CMAKE_SOURCE_DIR}, otherwise this is the top level directory of your build tree
    \item \inlinecode{CMAKE_SOURCE_DIR}: this is the directory, from which cmake was started, i.e. the top level source directory
    \item \inlinecode{EXECUTABLE_OUTPUT_PATH}: set this variable to specify a common place where CMake should put all executable files (instead of \inlinecode{CMAKE_CURRENT_BINARY_DIR}):
    \begin{lstlisting}
    SET(EXECUTABLE_OUTPUT_PATH ${PROJECT_BINARY_DIR}/bin)
    \end{lstlisting}
    \item \inlinecode{LIBRARY_OUTPUT_PATH}: set this variable to specify a common place where CMake should put all libraries (instead of \inlinecode{CMAKE_CURRENT_BINARY_DIR}):
    \begin{lstlisting}
    SET(LIBRARY_OUTPUT_PATH ${PROJECT_BINARY_DIR}/lib)
    \end{lstlisting}
    \item \inlinecode{PROJECT_NAME}: the name of the project set by \inlinecode{PROJECT()} command.
    \item \inlinecode{PROJECT_SOURCE_DIR}: contains the full path to the root of your project source directory, i.e. to the nearest directory where CMakeLists.txt contains the \inlinecode{PROJECT()} command Now, you have to compile the test.cpp. The way to do this task is too simple. Add the following line into your CMakeLists.txt:
    \begin{lstlisting}
    add_executable(hello ${PROJECT_SOURCE_DIR}/test.cpp)
    \end{lstlisting}
\end{itemize}
    



        



    




    





    




