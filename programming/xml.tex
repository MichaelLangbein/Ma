\section{xml}

XML is the de-facto standard language for specifying capabilities, interfaces etc.

\subsection{Structure}
XML consist of elements with attributes. The general name for anything in xml is a \emph{node} - a node is an element, an attribute or a text. 

\subsection{Namespaces: xmlns}
Because the naming of xml-elements and attributes can lead to conflicts, they are usually prefixed with a namespace \inlinecode{xmlns}. The id of the namespace need not resolve to a URL of a xsd. If it doesn't, you can still provide a hint for where to search for a schema using the extra-element \inlinecode{<namespacename>:schemaLocation="<schemaurl>"}. 

\subsection{Schema definitions: xsd}
With xsd we can consult an online-resource to make sure a given xml-file conforms to a specification.

\paragraph{Xsd's description of elements (aka. element type)} is used to describe what elements look like. 
Xsd defines simple and complex types. 
\begin{itemize}
    \item \inlinecode{<simpleType>} simple types can only have content directly contained between the element’s opening and closing tags. They cannot have attributes or child elements.
    \item \inlinecode{<complexType>} complex types can have attributes, can contain other elements, can contain a mixture of elements and text, etc etc.
\end{itemize}
Xsd also has its own, custom elements that are used to describe certain requirements.
\begin{itemize}
    \item \inlinecode{restriction} restricts an elements' content to a certain type (string, numeric, ...)
    \item \inlinecode{sequence}
\end{itemize} 

Here is an example from a simple xsd: 
\begin{lstlisting}[language=XML]
<element name="email">
    <simpleType>
      <restriction base="xsd:string" />
    </simpleType>
  </element>
  
  <element name="getPreferences">
    <complexType>
      <sequence>
        <element name="email" type="email" />
      </sequence>
    </complexType>
  </element>
\end{lstlisting}

\paragraph{Xsd's description of element's contents (aka. content type)} is used to describe what is expected to be inside an elements brackets. 
\begin{itemize}
    \item A simple type’s content can be one of:
        \begin{itemize}
            \item atomic types, which have indivisible values, such as \#000 and \#AACCDD
            \item list types, which have whitespace-separated lists of indivisible values, such as blue green red
            \item union types, which have either atomic or list values, but they can be the union of other types, such as blue \#000 red for a set of colors
        \end{itemize}
    \item Complex types have a “content model,” which refers to how the content (the data between the element’s opening and closing tags) is arranged:
        \begin{itemize}
            \item simple content is only character data, no child elements allowed
            \item element-only content is only children, no data allowed
            \item mixed content means character data and child elements can be intermingled
            \item empty content means the element is empty (<foo/>) and either conveys information by just existing, or has attributes but no content.
        \end{itemize}
\end{itemize}

Just to clarify: elements have an \emph{element type}, and their content has a \emph{content type}. By the way, attributes can only have simple types, because they cannot themselves have attributes or children.

\subsection{Marhsalling}

\paragraph{XML to Pojo}: JAXB
\begin{lstlisting}[language=xml]
  <jaxb:bindings
  xmlns:jaxb="http://java.sun.com/xml/ns/jaxb" 
  xmlns:xs="http://www.w3.org/2001/XMLSchema" 
  xmlns:xjc="http://java.sun.com/xml/ns/jaxb/xjc"  
  jaxb:version="2.0">

  <!-- <jaxb:bindings schemaLocation="http://www.w3.org/1999/xlink.xsd">
      <jaxb:bindings node="//xs:attributeGroup[@name='locatorAttrs']/xs:attribute[@ref=xlink:title]">
          <jaxb:property name="LinkTitle"></jaxb:property>
      </jaxb:bindings>
  </jaxb:bindings> -->

  <jaxb:bindings scd="x-schema::xlink" xmlns:xlink="http://www.w3.org/1999/xlink">
  <jaxb:bindings scd="/group::xlink:locatorModel/model::sequence/xlink:title">
    <jaxb:property name="LocatorTitle"/>
  </jaxb:bindings>
</jaxb:bindings>

</jaxb:bindings>
\end{lstlisting}

\paragraph{XML to JSON}: Jsonix
\href{https://github.com/highsource/jsonix-schema-compiler}{here}
