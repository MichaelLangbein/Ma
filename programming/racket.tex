\section{Racket}

The hallmark of the Lisp family is that programs are defined in terms of data structures rather than in terms of a text-based syntax. The most visible consequence is a rather peculiar visual aspect, which is dominated by parentheses. The more profound implication, and in fact the motivation for this uncommon choice, is the equivalence of code and data. Program execution in Lisp is nothing but interpretation of a data structure. It is possible, and common practice, to construct data structures programmatically and then evaluate them. The most frequent use of this characteristic is writing macros (which can be seen as code preprocessors) to effectively extend the language with new features. In that sense, all members of the Lisp family are “programmable programming languages”. 

\begin{lstlisting}[language=lisp]

; definitions are identical for variables and functions
(define name "Michael")
(define (add x y)
    (+ x y))

; use anonymous functions to create functions as return-values on the fly
(define (createPowerFunc pwr)
  (lambda (base)
    (expt base pwr)))
(define pwr4 (createPowerFunc 4))
(pwr4 3)
> 81


; conditionals
; if/else is simple:
(if (equal? name "Andreas")
      "Zieh ab!"
      "Servus!")
; if you want if/else with more than two cases, use cond:
(define (commentMyWeight w)
    (cond
      [(< w 50) "Gosh, you're a fether!"]
      [(< w 70) "Sounds fine."]
      [(< w 90) "You're a dude, right?"]
      [else "Seriously, thats too much!"])


; datatypes
; list
'("1" 2 c)
(list "a" "b")


\end{lstlisting}

There is a nice set of tutorials \href{https://beautifulracket.com/stacker/}{here}.
